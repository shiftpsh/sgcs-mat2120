\documentclass[runningheads,11pt]{../llncs}
\usepackage{geometry}
\usepackage{graphicx}
\usepackage{wrapfig}
\usepackage{import}
\usepackage{kotex}
\usepackage[dvipsnames]{xcolor}
\usepackage{fancyvrb}
\usepackage{listings}
\usepackage{indentfirst}
\usepackage{tabularx}
\usepackage{underscore}
\usepackage{multicol}
\usepackage{enumitem}
\usepackage{menukeys}
\usepackage{amsmath}
\usepackage{clrscode3e} % https://www.ctan.org/pkg/clrscode3e?lang=en
\usepackage[numbers,square,super]{natbib}
\usepackage{inconsolata} % Inconsolata
\usepackage{mathptmx} % Times New Roman
\usepackage{amsmath}
\usepackage{amssymb}
\newcommand{\divides}{\bigm|}
\newcommand{\ndivides}{%
  \mathrel{\mkern.5mu % small adjustment
    % superimpose \nmid to \big|
    \ooalign{\hidewidth$\big|$\hidewidth\cr$\nmid$\cr}%
  }%
}
\setmainfont{Times New Roman}
\setmonofont{Fira Code}
\usepackage{minted}
\lstset{basicstyle=\footnotesize\ttfamily,breaklines=true}
\renewcommand{\bibname}{참고문헌}
\setlength{\parindent}{1em}
\setlength{\parskip}{1em}
\linespread{1.2}
{\renewcommand{\arraystretch}{1.5}%
\setlength{\tabcolsep}{0.5em}%
\newenvironment{Figure}
  {\par\medskip\noindent\minipage{\linewidth}}
  {\endminipage\par\medskip}
\newcommand{\translation}[1]{\textsuperscript{#1}}
\newcommand{\complexity}[1]{$\mathcal{O}\left({#1}\right)$}

\makeatletter
\renewcommand\NAT@citesuper[3]{\ifNAT@swa
\if*#2*\else#2\NAT@spacechar\fi
\unskip\kern\p@\textsuperscript{\NAT@@open#1\if*#3*\else,\NAT@spacechar#3\fi\NAT@@close}%
   \else #1\fi\endgroup}
\makeatother

\definecolor{codehighlight}{RGB}{0,118,191}
\definecolor{codebg}{RGB}{250,250,250}

\setminted{fontsize=\footnotesize, 
		   linenos,
		   numbersep=8pt,
		   gobble=0,
		   frame=lines,
		   bgcolor=codebg,
       framesep=3mm,
       highlightcolor=codehighlight!15,
       mathescape=true} 

\begin{document}

\title{MAT2120 Number Theory\newline\space Problems IV}
\author{Suhyun Park (20181634)}
\institute{Department of Computer Science and Engineering, Sogang University}

\maketitle

% Question 1.
\subsubsection{1.} Let $d$ be a positive integer. Show that the simple continued fraction of $\sqrt{d^2+1}$ is
$\left[d;\,\overline{2d}\right]$, and find the simple continued fraction of $\sqrt{101}$.

\paragraph{Solution.} Since $\left\lfloor\sqrt{d^2+1}\right\rfloor=d$, the first term is given by $d$.

Subtracting $d$ from $\sqrt{d^2+1}$ gives
\begin{align*}
    \sqrt{d^2+1}-d &= \frac{\left(\sqrt{d^2+1}\right)^2-d^2}{\sqrt{d^2+1}+d} \\
    &= \frac{1}{\sqrt{d^2+1}+d},
\end{align*}
hence the second term is given by $2d$.

Repeating this process by subtracting $2d$ from $\sqrt{d^2+1}+d$ gives $\sqrt{d^2+1}-d$, which is same with above result;
thus $\sqrt{d^2+1}=\left[d;\,\overline{2d}\right]$, and therefore $\sqrt{101}=\left[10;\,\overline{20}\right]$.

% Question 2.
\subsubsection{2.} Show that the simple continued fraction of $\sqrt{d}$, where $d$ is a positive integer,
has period length 1 if and only if $d=a^2+1$, where $a$ is a nonnegative integer.

\begin{proof}
    ($\Rightarrow$) Suppose the period of the simple continued fraction of $\sqrt{d}$ is 1. Then we can see that
    $\sqrt{d} = \left[a;\,\overline{2a}\right]$.

    Let $x=\left[2a;\,\overline{2a}\right]$. Then $\left[a;\,\overline{2a}\right]=\left[a;\,x\right]$. Note that
    \[
        x=\left[ 2a;\,x \right] = 2a+\frac{1}{x},
    \]
    thus
    \begin{align*}
        x^2-2ax-1&=0\\
        \therefore x=a+\sqrt{a^2+1}.
    \end{align*}
    Hence
    \begin{align*}
        \sqrt{d} &= \left[a;\,x\right] \\
        &= a+\frac{1}{x} \\
        &= a+\frac{1}{a+\sqrt{a^2+1}} \\
        &= \sqrt{a^2+1} \\
        \therefore d&=a^2+1.
    \end{align*}

    ($\Leftarrow$) Proved in Problem 1.
\end{proof}

% Question 3.
\subsubsection{3.} Find the least positive solutions in integers of $x^2-29y^2=-1$.

\paragraph{Solution.} Note that $\sqrt{29}=\left[5;\,\overline{2,\,1,\,1,\,2,\,10}\right]$. The convergents $h_n$ and $k_n$ are

\begin{center}
    \begin{tabular}{c|ccccccccc}
    \hline
    $n$ & $-2$ & $-1$ & 0 & 1 & 2 & 3 & 4 & 5 & $\cdots$ \\
    \hline
    $a_n$ & -- & -- & 5 & 2  & 1  & 1  & 2  & 10 & $\cdots$ \\
    \hline
    $h_n$ & 0  & 1  & 5 & 11 & 16 & 27 & \textbf{70} & 727 & $\cdots$ \\
    $k_n$ & 1  & 0  & 1 & 2  & 3  & 5  & \textbf{13} & 135 & $\cdots$ \\
    \hline
    $h_n^2-29k_n^2$ & $-1$ & 1 & $-4$ & 5 & $-5$ & 4 & $\mathbf{-1}$ & 4 & $\cdots$ \\
    \hline
    \end{tabular}
\end{center}

Thus the minimal solution is given by $\left(x,\,y\right)=\left(70,\,13\right)$.

% Question 4.
\subsubsection{4.} Show that if $p$ is prime and $x^p+y^p=z^p$, then $p \divides \left(x+y-z\right)$.

\begin{proof}
    Recall that, by Fermat, for $\forall a \in \mathbb{Z}$:
    \[
        a^p \equiv a \pmod{p}.
    \]
    Thus
    \begin{align*}
        x^p+y^p&\equiv z^p \pmod{p} \\
        \Rightarrow x+y&\equiv z \pmod{p} \\
        \therefore p &\divides \left(x+y-z\right).
    \end{align*}
    \qed
\end{proof}

% Question 5.
\subsubsection{5.} Determine all right triangles with sides of integral length whose areas equal their perimeters.

\paragraph{Solution.} Let $a$ and $b$ be the lengths of the sides. We have the relation of
\[
    \frac{ab}{2}=a+b+\sqrt{a^2+b^2},
\]
hence
\begin{align*}
    ab &= 2a + 2b + 2\sqrt{a^2+b^2} \\
    \Rightarrow\left(ab-2a-2b\right)^2 &= 4a^2+4b^2 \\
    \Rightarrow a^2b^2-4a^2b-4ab^2+8ab &= 0 \\
    \Rightarrow ab - 4a - 4b + 8 &= 0 \qquad \because ab \neq 0 \\
    \Rightarrow a=4\cdot\frac{b-2}{b-4}.
\end{align*}
If $c:=b-4$, then
\[
    a=4+\frac{8}{c},
\]
thus if $a$ is an integer, then $c=\pm 1,\,\pm 2,\,\pm 4$ or $\pm 8$; hence
$b=2,\,3,\,5,\,6,\,8$ or 12. Calculating for each $b$ gives

\begin{center}
    \begin{tabular}{c|cccccc}
        \hline
        $b$ & 2 & 3 & 5 & 6 & 8 & 12 \\
        \hline
        $a$ & 0 & $-4$ & 12 & 8 & 6 & 5 \\
        \hline
    \end{tabular}
\end{center}

hence there exists only two triangles, $\left(3,\,4,\,5\right)$ and $\left(5,\,12,\,13\right)$, which its
areas equal their perimeters.

% Question 6.
\subsubsection{6.} Use the fact that 2 is not a congruent number to show that $\sqrt{2}$ is irrational.

\begin{proof}
    Suppose the right triangle with sides 2, 2, and $2\sqrt{2}$. Suppose $\sqrt{2}\in\mathbb{Q}$.
    then $2 \cdot 2 \cdot \frac{1}{2} = 2$ is congruent, but it is not, hence $\sqrt{2}\not\in\mathbb{Q}$. \qed
\end{proof}


% Question 7.
\subsubsection{7.} Show that if $\left(x,\,y,\,z\right)$ is a Pythagorean triple, then $xyz$ is divisible by 60.

\begin{proof}
    Since $\left(x,\,y,\,z\right)$ is a Pythagorean triple, for $r,\,s\in\mathbb{Z}^+$ and $r+s \equiv 1 \pmod{2}$, let
    \[
        x=r^2-s^2 \quad y=2rs \quad z=r^2+s^2,
    \]
    which gives
    \[
        xyz= 2rs\left(r^2-s^2\right)\left(r^2+s^2\right).
    \]

    To prove that $60 \divides xyz$, it suffices to prove that $3\divides xyz$, $4\divides xyz$ and $5\divides xyz$.

    \begin{itemize}
        \item[($4\divides xyz$)] Since $r+s \equiv 1 \pmod{2}$, $r$ or $s$ is even; therefore $4 \divides 2rs \Rightarrow 4 \divides xyz$.
        \item[($3\divides xyz$)] If $3\divides r$, it is trivial.
        Otherwise if $3 \ndivides r$ and $3 \ndivides s$, by Euler,
        $r^{\phi\left(3\right)} \equiv 1 \pmod{3} \Rightarrow r^2-1 \equiv 0 \pmod{3}$ and $s^2-1 \equiv 0 \pmod{3}$;
        hence $3 \divides \left[\left(r^2-1\right)-\left(s^2-1\right)\right]=\left(r^2-s^2\right)$.
        \item[($5\divides xyz$)] Similarity, if $5\divides r$, it is trivial.
        Otherwise if $5 \ndivides r$, by Euler, $5 \divides \left(r^4-s^4\right)$. \qed
    \end{itemize}
\end{proof}

\end{document}

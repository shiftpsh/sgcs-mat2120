\documentclass[runningheads,11pt]{../llncs}
\usepackage{geometry}
\usepackage{graphicx}
\usepackage{wrapfig}
\usepackage{import}
\usepackage{kotex}
\usepackage[dvipsnames]{xcolor}
\usepackage{fancyvrb}
\usepackage{listings}
\usepackage{indentfirst}
\usepackage{tabularx}
\usepackage{underscore}
\usepackage{multicol}
\usepackage{enumitem}
\usepackage{menukeys}
\usepackage{amsmath}
\usepackage{clrscode3e} % https://www.ctan.org/pkg/clrscode3e?lang=en
\usepackage[numbers,square,super]{natbib}
\usepackage{inconsolata} % Inconsolata
\usepackage{mathptmx} % Times New Roman
\usepackage{amsmath}
\usepackage{amssymb}
\newcommand{\divides}{\bigm|}
\newcommand{\ndivides}{%
  \mathrel{\mkern.5mu % small adjustment
    % superimpose \nmid to \big|
    \ooalign{\hidewidth$\big|$\hidewidth\cr$\nmid$\cr}%
  }%
}
\setmainfont{Times New Roman}
\setmonofont{Fira Code}
\usepackage{minted}
\lstset{basicstyle=\footnotesize\ttfamily,breaklines=true}
\renewcommand{\bibname}{참고문헌}
\setlength{\parindent}{1em}
\setlength{\parskip}{1em}
\linespread{1.2}
{\renewcommand{\arraystretch}{1.5}%
\setlength{\tabcolsep}{0.5em}%
\newenvironment{Figure}
  {\par\medskip\noindent\minipage{\linewidth}}
  {\endminipage\par\medskip}
\newcommand{\translation}[1]{\textsuperscript{#1}}
\newcommand{\complexity}[1]{$\mathcal{O}\left({#1}\right)$}

\makeatletter
\renewcommand\NAT@citesuper[3]{\ifNAT@swa
\if*#2*\else#2\NAT@spacechar\fi
\unskip\kern\p@\textsuperscript{\NAT@@open#1\if*#3*\else,\NAT@spacechar#3\fi\NAT@@close}%
   \else #1\fi\endgroup}
\makeatother

\definecolor{codehighlight}{RGB}{0,118,191}
\definecolor{codebg}{RGB}{250,250,250}

\setminted{fontsize=\footnotesize, 
		   linenos,
		   numbersep=8pt,
		   gobble=0,
		   frame=lines,
		   bgcolor=codebg,
       framesep=3mm,
       highlightcolor=codehighlight!15,
       mathescape=true} 

\begin{document}

\title{MAT2120 Number Theory\newline\space Problems III}
\author{Suhyun Park (20181634)}
\institute{Department of Computer Science and Engineering, Sogang University}

\maketitle

% Question 1.
\subsubsection{1.} Solve the congrugence $8x^5 \equiv 3 \pmod{13}$ using the following
table of indices for the prime 13 relative to the primitive root 2:
\begin{center}
    \begin{tabular}{|c||c|c|c|c|c|c|c|c|c|c|c|c|}
        \hline
        $a$ & 1 & 2 & 3 & 4 & 5 & 6 & 7 & 8 & 9 & 10 & 11 & 12 \\
        \hline
        $\ind_2 a$ & 12 & 1 & 4 & 2 & 9 & 5 & 11 & 3 & 8 & 10 & 7 & 6 \\
        \hline
    \end{tabular}
\end{center}

\paragraph{Solution.}
Since $8x^5 \equiv 3 \pmod{13}$, $x^5 \equiv 2 \pmod{13}$.

Taking $\ind_2$ on $x^5 \equiv 2 \pmod{13}$ gives
\begin{align*}
    \ind_2 x^5 &\equiv \ind_2 2 \pmod{12} \\
    \Leftrightarrow 5 \ind_2 x &\equiv 1 \pmod{12} \\
    \Leftrightarrow \ind_2 x &\equiv 5 \pmod{12},
\end{align*}
therefore $x \equiv 6 \pmod{13}$.

% Question 2.
\subsubsection{2.} Show that if $p$ is a prime of the form $4k+1$, then
$\legendre{1}{p} + \legendre{2}{p} + \cdots + \legendre{q}{p}=0$, where
$q=\frac{p-1}{2}$.

\begin{proof}
    Recall that, if $p$ is odd prime, then
    \[
        \legendre{-1}{p} = 1 \qquad \mbox{if and only if } p \equiv 1 \pmod{4}.
    \]
    Note that
    \[
        \legendre{p-a}{p}=\legendre{-a}{p}=\legendre{-1}{p}\legendre{a}{p}=\legendre{a}{p}.
    \]
    \underline{Claim} $p$ : odd prime, then $\sum_{k=1}^{p-1}\legendre{k}{p}=0$. Thus,
    \begin{align*}
        0 &= \legendre{1}{p}+\legendre{2}{p}+\cdots+\legendre{p-1}{p} \\
        &= 2\left[\legendre{1}{p}+\legendre{2}{p}+\cdots+\legendre{q}{p}\right] \qquad \mbox{where } q=\frac{p-1}{2}.
    \end{align*}
\end{proof}

% Question 3.
\subsubsection{3.} Let $p$ be an odd prime. Show that 2 is a quadratic residue
of $p$ if $p \equiv \pm 1 \pmod{8}$ and a quadratic nonresidue of $p$ if
$p \equiv \pm 3 \pmod{8}$.

\begin{proof}
    By Gauss's Lemma, $\legendre{2}{p}=\left(-1\right)^s$
    where $s$ is the number of least positive residues mod $p$ of the integers $1\cdot 2$,
    $2 \cdot 2$, $3 \cdot 2$, $\cdots$, $\frac{p-1}{2}\times 2$ that are greater than $\frac{p}{2}$.  
    
    Since all of these integers are less than $p$, we need only count these greater than $\frac{p}{2}$ to find.
    Note that the integers $2j$ where $1\leq j\leq \frac{p-1}{2}$ are less than $\frac{p}{2}$ whem
    $j \leq \frac{p}{4}$.

    Thus $\exists \left\lfloor\frac{p}{4}\right\rfloor$ integers in the set less
    that $\frac{p}{2}$. By Gauss's lemma,
    \[
        \legendre{2}{p} = \left(-1\right)^{\frac{p-1}{2} - \left\lfloor\frac{p}{4}\right\rfloor}.
    \]
    To prove the theorem, it suffice to show that $\frac{p-1}{2} - \left\lfloor\frac{p}{4}\right\rfloor \equiv \frac{p^2-1}{8} \pmod{2}$
    for every odd integer $p$.

    Note that it holds for a positive integer $p$ if and only if it holds for $p+8$.
    It can be checked that it holds for $p\equiv\pm 1$, $p\equiv\pm 3 \pmod{8}$.
    Hence we conclude that it holds for every odd integer $p$.
\end{proof}

% Question 4.
\subsubsection{4.} Using problem 2 above, show that the prime 1999
divides $2^{999}-1$.

\begin{proof}
    By Problem 3, 2 is a quadratic residue of 1999; hence there exists integer $x$
    such that $x^2 \equiv 2 \pmod{1999}$, hence
    \begin{align*}
        x^2 &\equiv 2 \pmod{1999} \\
        \Rightarrow \left(x^2\right)^{999} &\equiv 2^{999} \pmod{1999} \\
        \Rightarrow x^{1998} = x^{\phi\left(1999\right)} &\equiv 2^{999} \pmod{1999} \\
        \Rightarrow 1 &\equiv 2^{999} \pmod{1999} \\
        \Rightarrow 0 &\equiv 2^{999}-1 \pmod{1999}.
    \end{align*}
    Therefore 1999 divides $2^{999}-1$.
\end{proof}

% Question 5.
\subsubsection{5.} Calculate $\legendre{6}{19}$ using \textbf{(a)} Euler's
criterion \textbf{(b)} Gauss's Lemma \textbf{(c)} the law of quadratic
reciprocity.

\paragraph{(a)} By Euler's criterion,
\begin{align*}
    \legendre{6}{19} & \equiv 6^{\frac{19-1}{2}} \equiv 6^9 \pmod{19} \\
    &\equiv 6 \cdot \left(6^2\right)^4 \equiv 6 \cdot 36^4 \pmod{19} \\
    &\equiv 6 \cdot \left(-2\right)^4 \equiv 6 \cdot 16 \pmod{19} \\
    &\equiv 6 \cdot \left(-3\right) \equiv -18 \pmod{19} \\
    &\equiv 1 \pmod{19},
\end{align*}
hence $\legendre{6}{19}=1$.

\paragraph{(b)} By Gauss's Lemma,
\[
    \legendre{6}{19} = \left(-1\right)^s
\]
where $s$ is the number of least positive integers mod 19 of the
integers $1\cdot 6,\,2\cdot 6,\,3\cdot 6,\,\cdots,\,\frac{19-1}{2}\cdot 6$
that are greater than $\frac{19}{2}$.
Note that
\begin{align*}
    & \left\{ 1\cdot 6,\,2\cdot 6,\,3\cdot 6,\,4\cdot 6,\,
    5\cdot 6,\,6\cdot 6,\,7\cdot 6,\,8\cdot 6,\,9\cdot 6\right\} \\
    &= \left\{ 6,\,12,\,18,\,24,\,30,\,36,\,42,\,48,\,54\right\} \\
    &\equiv \left\{ 6,\,12,\,18,\,5,\,11,\,17,\,4,\,10,\,16 \right\} \pmod{19},
\end{align*}
hence $s=6 \Rightarrow \legendre{6}{19}=\left(-1\right)^6=1$.

\paragraph{(c)} By the law of quadratic reciprocity,

\begin{align*}
    \legendre{6}{19} &= \legendre{2}{19}\legendre{3}{19} \\
    &= \left(-1\right)^{\frac{19^2-1}{8}} \times \left(-1\right)^{\frac{3-1}{2}\frac{19-1}{2}}\legendre{19}{3} \\
    &= \left(-1\right)^{45} \times \left( -1 \right)^{9} \legendre{1}{3} \\
    &= \left(-1\right) \times \left(-1\right) \times 1 \\
    &= 1.
\end{align*}

% Question 6.
\subsubsection{6.} Let $F_n$ denote the $n$-th Fibonacci number,
and $p$ an odd prime with $p \neq 5$. Show that
\[
    F_p \equiv \begin{cases}
        1 \pmod{p} & \mbox{if } p \equiv \pm 1 \pmod{5},\\
        -1 \pmod{p} & \mbox{if } p \equiv \pm 2 \pmod{5}.
    \end{cases}
\]

\begin{proof}
    Note that
    \begin{align*}
        \sqrt{5} F_p &= \left(\frac{1+\sqrt{5}}{2}\right)^p - \left(\frac{1-\sqrt{5}}{2}\right)^p \\
        \Rightarrow \sqrt{5} 2^pf_p &= \left(1+\sqrt{5}\right)^p - \left(1-\sqrt{5}\right)^p \\
        &= \sum_{r=0}^p \binom{p}{r}\sqrt{5}^r - \sum_{r=0}^p \binom{p}{r}\left(-\sqrt{5}\right)^r \\
        &= 2\left[\binom{p}{1}\sqrt{5} + \binom{p}{3}\sqrt{5}^3 + \cdots + \binom{p}{p}\sqrt{5}^p\right],
    \end{align*}
    thus
    \[
        2^{p-1}F_p = \binom{p}{1} + 5\binom{p}{3} + \cdots + 5^{\frac{p-1}{2}}\binom{p}{p}.
    \]
    Since $p \divides \binom{p}{k}$ for $1 \leq k \leq p-1$ and $\binom{p}{p}=1$, we can see that
    \[
        2^{p-1}F_p \equiv 5^{\frac{p-1}{2}} \pmod{p}.
    \]
    Note that, by Fermat, $2^{p-1} \equiv 1 \pmod{p}$ and by Euler, $5^{\frac{p-1}{2}}\equiv\legendre{5}{p} \pmod{p}$,
    hence $F_p \equiv \legendre{5}{p} \pmod{p}$. Note also that, by the quadratic reciprocity, $\legendre{5}{p}=\legendre{p}{5}$.
    Hence
    \[
        F_p \equiv \legendre{5}{p} = \legendre{p}{5} = \begin{cases}
            1 \pmod{p} & p = \pm 1 \pmod{5}, \\
            -1 \pmod{p} & p = \pm 2 \pmod{5}.
        \end{cases}
    \]
\end{proof}

% Question 7.
\subsubsection{7.} Show that if $p>3$ is an odd prime, then
\[
    \legendre{3}{p} = \begin{cases}
        1 & \mbox{if } p \equiv \pm 1 \pmod{12}, \\
        -1 & \mbox{if }p \equiv \pm 5 \pmod{12}.
    \end{cases}
\]
\begin{proof}
    If $p \equiv 1 \pmod{4}$,
    \[
        \legendre{3}{p}=\legendre{p}{3}=\begin{cases}
            1 & \mbox{if } p \equiv 1 \pmod{3}, \\
            -1 & \mbox{if } p \equiv 2 \pmod{3}. \\
        \end{cases}
    \]
    Similarity, if $p \equiv -1 \pmod{4}$,
    \[
        \legendre{3}{p}=-\legendre{p}{3}=\begin{cases}
            -1 & \mbox{if } p \equiv 1 \pmod{3}, \\
            1 & \mbox{if } p \equiv 2 \pmod{3}. \\
        \end{cases}
    \]
    Hence it is clear that
    \[
        \legendre{3}{p} = \begin{cases}
            1 & \mbox{if } p \equiv \pm 1 \pmod{12}, \\
            -1 & \mbox{if }p \equiv \pm 5 \pmod{12}.
        \end{cases}
    \]
\end{proof}

% Question 8.
\subsubsection{8.} Show that if $p>3$ is an odd prime, then
\[
    \legendre{-3}{p} = \begin{cases}
        1 & \mbox{if } p \equiv 1 \pmod{6}, \\
        -1 & \mbox{if }p \equiv -1 \pmod{6}.
    \end{cases}
\]
\begin{proof}
    Note that
    \begin{align*}
        \legendre{-3}{p} &= \legendre{-1}{p}\legendre{3}{p} \\
        &= \left(-1\right)^{\frac{p-1}{2}}\left(-1\right)^{\frac{3-1}{2}\frac{p-1}{2}}\legendre{p}{3} \\
        &= \legendre{p}{3}.
    \end{align*}
    Hence, if $p \equiv 1 \pmod{6}$, then $\legendre{-3}{p}=\legendre{p}{3}=\legendre{1}{3}=1$,
    and if $p \equiv -1 \pmod{6}$, then $\legendre{-3}{p}=\legendre{p}{3}=\legendre{-1}{3}=-1$.
\end{proof}

\end{document}

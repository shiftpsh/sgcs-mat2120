\documentclass[runningheads,11pt]{../llncs}
\usepackage{geometry}
\usepackage{graphicx}
\usepackage{wrapfig}
\usepackage{import}
\usepackage{kotex}
\usepackage[dvipsnames]{xcolor}
\usepackage{fancyvrb}
\usepackage{listings}
\usepackage{indentfirst}
\usepackage{tabularx}
\usepackage{underscore}
\usepackage{multicol}
\usepackage{enumitem}
\usepackage{menukeys}
\usepackage{amsmath}
\usepackage{clrscode3e} % https://www.ctan.org/pkg/clrscode3e?lang=en
\usepackage[numbers,square,super]{natbib}
\usepackage{inconsolata} % Inconsolata
\usepackage{mathptmx} % Times New Roman
\usepackage{amsmath}
\usepackage{amssymb}
\newcommand{\divides}{\bigm|}
\newcommand{\ndivides}{%
  \mathrel{\mkern.5mu % small adjustment
    % superimpose \nmid to \big|
    \ooalign{\hidewidth$\big|$\hidewidth\cr$\nmid$\cr}%
  }%
}
\setmainfont{Times New Roman}
\setmonofont{Fira Code}
\usepackage{minted}
\lstset{basicstyle=\footnotesize\ttfamily,breaklines=true}
\renewcommand{\bibname}{참고문헌}
\setlength{\parindent}{1em}
\setlength{\parskip}{1em}
\linespread{1.2}
{\renewcommand{\arraystretch}{1.5}%
\setlength{\tabcolsep}{0.5em}%
\newenvironment{Figure}
  {\par\medskip\noindent\minipage{\linewidth}}
  {\endminipage\par\medskip}
\newcommand{\translation}[1]{\textsuperscript{#1}}
\newcommand{\complexity}[1]{$\mathcal{O}\left({#1}\right)$}

\makeatletter
\renewcommand\NAT@citesuper[3]{\ifNAT@swa
\if*#2*\else#2\NAT@spacechar\fi
\unskip\kern\p@\textsuperscript{\NAT@@open#1\if*#3*\else,\NAT@spacechar#3\fi\NAT@@close}%
   \else #1\fi\endgroup}
\makeatother

\definecolor{codehighlight}{RGB}{0,118,191}
\definecolor{codebg}{RGB}{250,250,250}

\setminted{fontsize=\footnotesize, 
		   linenos,
		   numbersep=8pt,
		   gobble=0,
		   frame=lines,
		   bgcolor=codebg,
       framesep=3mm,
       highlightcolor=codehighlight!15,
       mathescape=true} 

\begin{document}

\title{MAT2120 Number Theory\newline\space Problems I}
\author{Suhyun Park (20181634)}
\institute{Department of Computer Science and Engineering, Sogang University}

\maketitle

% Question 1.
\subsubsection{1.} Show that $\left(a,\,c\right)=1$, $\left(b,\,c\right)=1$
if and only if $\left(ab,\,c\right)=1$.

\begin{proof}
    ($\Rightarrow$) Suppose $\left(ab,\,c\right)=d>1$. Then
    \[
        d \divides c,\, d \divides ab.    
    \]
    For all integers $d_1$, $d_2$ such that $d_1 \divides a$, $d_2 \divides b$
    and $d=d_1d_2$,
    \[
        d_1 \divides c,\, d_2\divides c \Rightarrow \left(a,\,c\right) = d_1,\, \left(b,\,c\right) = d_2.    
    \]
    Since $d_1d_2 = d > 1$, $d_1 > 1$ or $d_2 > 1$. This is a contradiction;
    thus $\left(ab,\,c\right)=1$ if $\left(a,\,c\right)=1$, $\left(b,\,c\right)=1$.

    ($\Leftarrow$) Suppose $\left(a,\,c\right)=d>1$. Then $d \divides ab$, thus
    $\left(ab,\,c\right)=d>1$, which contradicts. Similarily when if
    $\left(b,\,c\right)=d>1$, $\left(ab,\,c\right)=d>1$. Thus $\left(a,\,c\right)=1$, $\left(b,\,c\right)=1$
    if $\left(ab,\,c\right)=1$. \qed
\end{proof}

% Question 2.
\subsubsection{2.} If $\left(a,\,b\right)=1$, prove that $\left(a+b,\,a-b\right)=1$ or 2.

\begin{proof}
    Since $\left(a,\,b\right)=1$, $\left(2a,\,2b\right)=2$.

    Let $\left(a+b,\,a-b\right)=d$. Then
    \begin{align*}
        & d \divides \left(a+b\right),\, d\divides \left(a-b\right) \\
        \Rightarrow & d \divides \left[\left(a+b\right)+\left(a-b\right)\right],\,
        d \divides \left[\left(a+b\right)-\left(a-b\right)\right] \\
        \Rightarrow & d \divides 2a,\, d\divides 2b
    \end{align*}

    Thus $d = 1$ or 2, given the fact that $\left(2a,\,2b\right)=2$. \qed
\end{proof}

% Question 3.
\subsubsection{3.} Let $\left(a,\,b\right)=10$.
Find all possible values of $\left(a^3,\,b^4\right)$.

\begin{lemma}
    If $\left(x,\,y\right)=1$, $\left(x^a,\,y^b\right)=1$ for nonnegative integers
    $a$ and $b$.
\end{lemma}

\begin{proof}
    If the set of prime factors of $x$ is $P = \left\{p_1,\,p_2,\,\cdots,\,p_n\right\}$,
    and that of $y$ is $Q = \left\{q_1,\,q_2,\,\cdots,\,q_m\right\}$, i. e. 
    \begin{align*}
        x&= \prod_{k=1}^n p_k^{e_k} \\
        y&= \prod_{k=1}^m q_k^{e^\prime_k}
    \end{align*}
    for $e_i,\, e^\prime_i \in \mathbb{Z}^+$, then clearly if and only if
    $P \cap Q = \emptyset$, then $\left(x,\,y\right)=1$.
    
    Since
    \begin{align*}
        x^a&= \left[\prod_{k=1}^n p_k^{e_k}\right]^a=\prod_{k=1}^n p_k^{ae_k} \\
        y^b&= \left[\prod_{k=1}^m q_k^{e^\prime_k}\right]^b=\prod_{k=1}^m q_k^{be^\prime_k},
    \end{align*}
    the set of prime factors of $x^a$ is also $P$, and that of $y^b$ is also $Q$. Since $P \cap Q = \emptyset$, 
    $\left(x^a,\,y^b\right)=1$. \qed
\end{proof}

Since $\left(a,\,b\right)=10$, $10\divides a$, $10\divides b$.
We let $a=10a_0$ and $b=10b_0$, hence $\left(a_0,\,b_0\right)=1$. Then
\begin{align*}
    \left(a^3,\,b^4\right) &= \left(10^3a_0^3,\,10^4b_0^4\right) \\
    &= 10^3 \left(a_0^3,\, 10b_0^4\right) 
\end{align*}

Suppose $\left(a_0^3,\, 10b_0^4\right) =k$, which $k \divides a_0^3$
and $k \divides 10b_0^4$. Note that $\left(a_0^3,\,b_0^4\right)=1$ by Lemma and $k \divides a_0^3$,
hence $k \ndivides b_0^4$, thus $k \divides 10$.

Since $a$ and $b$ are nonnegative, possible values for $k$ is 1, 2, 5, and 10. Thus, possible values
of $\left(a^3,\,b^4\right) = 10^3k$ is $10^3$, $2 \cdot 10^3$, $5 \cdot 10^3$, and $10^4$.

% Question 4.
\subsubsection{4.} Show that $e=\sum_{n=0}^\infty \frac{1}{n!}$ is irrational.
(Hint. Suppose $e=\frac{p}{q}$ with positive integers $p$ and $q$.
Show that $q!e$ and $q!\sum_{n=0}^q \frac{1}{n!}$ are both integers.)

\begin{proof}
    Suppose that $e=\sum_{n=0}^\infty \frac{1}{n!}$ is rational. i. e.,
    there exists positive integers $p$ and $q$ such that $e=\frac{p}{q}$.

    Then $q!e = \left(q-1\right)!qe = \left(q-1\right)!p$ is an integer.
    Also,
    \begin{align*}
        q! \sum_{n=0}^q \frac{1}{n!} &= \sum_{n=0}^q \frac{q!}{n!} \\
        &= \sum_{n=0}^q \prod_{k=n+1}^q k
    \end{align*}
    is an integer. Thus,
    \begin{align*}
        q!e - q!\sum_{n=0}^q \frac{1}{n!}
        &= q!\sum_{n=0}^\infty \frac{1}{n!} - q!\sum_{n=0}^q \frac{1}{n!} \\
        &= q!\sum_{n=q+1}^\infty \frac{1}{n!}
    \end{align*}
    should be also an (positive) integer.

    Since
    \[
        \frac{q!}{n!} = \frac{1}{\prod_{k=q+1}^n k} \leq \frac{1}{\left(q+1\right)^{n-q}}
    \]
    is strict for every $n \geq b+2$, we can conclude that
    \begin{align*}
        q!\sum_{n=q+1}^\infty \frac{1}{n!} &< \sum_{n=q+1}^\infty \frac{1}{\left(q+1\right)^{n-q}} \\
        &= \sum_{k=1}^\infty \frac{1}{\left(q+1\right)^k} \\
        &= \frac{1}{q}.
    \end{align*}

    Note that $q!\sum_{n=q+1}^\infty \frac{1}{n!}$ should be a positive integer,
    but it is impossible that a positive integer is less than $\frac{1}{q}$, given the fact that $q$ is also
    a positive integer. Hence $e$ is not rational, thus irrational.
\end{proof}

% Question 5.
\subsubsection{5.} Show that if $k$ is an integer, then the integers
$6k-1$, $6k+1$, $6k+2$, $6k+3$, and $6k+5$ are pairwise relatively prime.

\begin{proof}
    If $\left(a,\,b\right)=d$, then $d \divides a$ and $d \divides b$;
    giving that $d \divides \left(a-b\right)$.

    Hence for some integer $k$, if $a+k=b$ and $\left(a,\,k\right)=1$
    and $\left(b,\,k\right)=1$,
    then $\left(a,\,b\right)=1$ because $\left(a,\,b\right) \divides \left(b-a\right) \Rightarrow \left(a,\,b\right) \divides k$,
    giving that $\left(a,\,b\right)$ is a common divisor of $a$ and $k$.

    For $k=1$,
    $\left(6k+1,\,6k+2\right)=1$
    and $\left(6k+2,\,6k+3\right)=1$.

    For $k=2$,
    $\left(6k-1,\,6k+1\right)=1$,
    $\left(6k+1,\,6k+3\right)=1$,
    and $\left(6k+3,\,6k+5\right)=1$.

    For $k=3$,
    $\left(6k-1,\,6k+2\right)=1$,
    and $\left(6k+2,\,6k+5\right)=1$.

    For $k=4$,
    $\left(6k-1,\,6k+3\right)=1$,
    and $\left(6k+1,\,6k+5\right)=1$.

    For $k=6$,
    $\left(6k-1,\,6k+5\right)=1$.

    Hence the integers
    $6k-1$, $6k+1$, $6k+2$, $6k+3$, and $6k+5$ are pairwise relatively prime. \qed
\end{proof}

% Question 6.
\subsubsection{6.} Show that if $a$ and $p$ are positive integers such that $a^p-1$
is prime, then $a=2$ or $p=1$.

\begin{proof}
    Note that
    \[
        a^p-1=\sum_{r=0}^{p-1}\left(a^{r+1}-a^r\right)=\left(a-1\right)\sum_{r=0}^{p-1} a^r.
    \]
    Thus $\left(a-1\right) \divides \left(a^p-1\right)$ and $\left(\sum_{r=0}^{p-1} a^r\right) \divides \left(a^p-1\right)$.

    Suppose that $a^p-1$ is prime.
    \begin{enumerate}
        \item If $a < 2$, then $a=1 \Rightarrow 1^p-1=0$ is not prime.
        
        \item If $a > 2$, then $a-1 \geq 2$.
        
        Since $a^p-1$ is prime, the only prime
        factor for $a^p-1$ has to be $a-1$, implying that $\sum_{r=0}^{p-1} a^r=1 \Rightarrow p=1$.
        
        \item If $p > 1$, then $\sum_{r=0}^{p-1} a^r \geq \sum_{r=0}^{2-1} a^r= 1+a \geq 2$.
        
        Since $a^p-1$ is prime, the only prime factor for $a^p-1$ has to be $\sum_{r=0}^{p-1} a^r$,
        implying that $a-1=1 \Rightarrow a=2$.
    \end{enumerate}
    Hence if $a^p-1$ is prime, then $a=2$ or $p=1$. \qed
\end{proof}

% Question 7.
\subsubsection{7.} Show that if $2^p-1$ is prime, then $p$ is prime.

\begin{proof}
    Note that
    \[
        2^p-1 = \sum_{k=0}^{p-1} 2^k.    
    \]
    
    Suppose $p$ is not prime, giving the fact that there exists some integers
    $p_1,\,p_2 \geq 2$ such that $p=p_1p_2$. Then
    \begin{align*}
        2^p-1 &= \sum_{k=0}^{p-1} 2^k \\
        &= \sum_{k=0\cdot p_1}^{1\cdot p_1-1} 2^k
        + \sum_{k=1\cdot p_1}^{2\cdot p_1-1} 2^k
        + \cdots
        + \sum_{k=\left(p_2-2\right) p_1}^{\left(p_2-1\right) p_1-1} 2^k
        + \sum_{k=\left(p_2-1\right) p_1}^{p_2p_1-1} 2^k \\
        &= \left(2^{p_1}\right)^0 \sum_{k=0}^{p_1-1} 2^k
        + \left(2^{p_1}\right)^1 \sum_{k=0}^{p_1-1} 2^k
        + \cdots
        + \left(2^{p_1}\right)^{p_2-2} \sum_{k=0}^{p_1-1} 2^k
        + \left(2^{p_1}\right)^{p_2-1} \sum_{k=0}^{p_1-1} 2^k \\
        &= \left[\sum_{k=0}^{p_2-1} \left(2^{p_1}\right)^k\right]
        \left[\sum_{k=0}^{p_1-1} 2^k\right]
    \end{align*}
    which $2^p-1$ is clearly not prime. Hence if $2^p-1$ is prime, then $p$ is also prime.
    \qed
\end{proof}

% Question 8.
\subsubsection{8.} Show that if $a$ is a positive integer and $a^m+1$ is an odd
prime, then $m=2^n$ for some nonnegative integer $n$.

\begin{proof}
    Suppose that $a^m+1$ is prime and $m\neq 2^n$ for any integer $n$. Then
    $m$ can be expressed as $m=rs$, where $1\leq r,\,s < m$, and $s$ is odd.

    Note that for any $l \in \mathbb{Z}^+$, 
    \[
        \left(x-y\right) \divides \left(x^l-y^l\right).
    \]

    Put $x=a^r$ and $y=-1$. then
    \begin{align*}
        & \left(a^r-1\right) \divides \left[\left(a^r\right)^s - \left(-1\right)^s\right] \\
        \Rightarrow & \left(a^r-1\right) \divides \left(a^{rs} + 1\right) \\
        \Rightarrow & \left(a^r-1\right) \divides \left(a^m+1\right).
    \end{align*}

    Hence if $m \leq 2^n$, $a^m+1$ is clearly not prime; thus
    if $a^m+1$ is prime then $m=2^n$ for some nonnegative integer $n$.
\end{proof}

% Question 9.
\subsubsection{9.} Show that if $a$ and $b$ are positive integers and if
$a^3 \divides b^2$, then $a \divides b$.

Let
\begin{align*}
    a &= p_1^{a_1} p_2^{a_2} \times \cdots \times p_{n-1}^{a_{n-1}} p_n^{a_n} \\
    b &= p_1^{b_1} p_2^{b_2} \times \cdots \times p_{n-1}^{b_{n-1}} p_n^{b_n}.
\end{align*}
where $a_i,\,b_i \in \mathbb{Z}^+ \cup \left\{0\right\}$ and $p_i$ is prime for $1 \leq i \leq n$.

Then if $a^3 \divides b^2$, it is clear that
\begin{align*}
    & \left(p_1^{a_1} p_2^{a_2} \times \cdots \times p_{n-1}^{a_{n-1}} p_n^{a_n}\right)^3
    \divides \left(p_1^{b_1} p_2^{b_2} \times \cdots \times p_{n-1}^{b_{n-1}} p_n^{b_n}\right)^2 \\
    \Rightarrow & \left(p_1^{3a_1} p_2^{3a_2} \times \cdots \times p_{n-1}^{3a_{n-1}} p_n^{3a_n}\right)
    \divides \left(p_1^{2b_1} p_2^{2b_2} \times \cdots \times p_{n-1}^{2b_{n-1}} p_n^{2b_n}\right).
\end{align*}
Since $p_1,\, p_2,\, \cdots,\, p_n$ are primes,
\begin{align*}
    & 3a_i \leq 2b_i \\
    \Rightarrow & a_i \leq b_i \qquad \mbox{for all } 1 \leq i \leq n.
\end{align*}
Hence
\begin{align*}
    & \left(p_1^{a_1} p_2^{a_2} \times \cdots \times p_{n-1}^{a_{n-1}} p_n^{a_n}\right)
    \divides \left(p_1^{b_1} p_2^{b_2} \times \cdots \times p_{n-1}^{b_{n-1}} p_n^{b_n}\right) \\
    \Rightarrow & a \divides b.
\end{align*}
\qed

% Question 10.
\subsubsection{10.} Find all integer solutions of the following system of
Diophantine equations:
\[
\begin{cases}
    x+y+z=100\\
    x+8y+50z=156
\end{cases}
\]

Subtracting the first equation from the second equation, we get
\[
    7y + 49z = 56 \Rightarrow y + 7z = 8.
\]
Then the arbitary solution for $y$ and $z$ is given by $y_0=1,\,x_0=1$,
and the general solution exists as

\[
    \left\{y,\,z\right\}
    = \left\{y_0 + \frac{7}{\left(1,\,7\right)}t,\, z_0 - \frac{1}{\left(1,\,7\right)}t\right\}
    = \left\{1 + 7t,\, 1 - t\right\}
\]
for any integer $t$.

Given the fact that $y+z=\left(1+7t\right)+\left(1-t\right)=2+6t$,
\[
    x+\left(y+z\right)=100 \Rightarrow x+2+6t=100 \Rightarrow x+6t=98.  
\]
Then the arbitary solution for $x$ and $t$ is given by $x_0=98,\,t_0=0$,
and the general solution exists as

\[
    \left\{x,\,t\right\}
    = \left\{x_0 + \frac{6}{\left(1,\,6\right)}u,\, t_0 + \frac{1}{\left(1,\,6\right)}u\right\}
    = \left\{98 + 6u,\, -u\right\}
\]
for any integer $u$.

Thus the general solution for $\left\{x,\,y,\,z\right\}$ is
\begin{align*}
    \left\{x,\,y,\,z\right\} &= \left\{98+6u, 1+7t,\,1-t\right\} \\
    &= \left\{98+6u, 1-7u,\,1+u\right\}
\end{align*}
for any integer $u$.

% Question 11.
\subsubsection{11.} What is the smallest positive rational number that can be expressed
in the form of $\frac{x}{30}+ \frac{y}{36}$ with integers $x$ and $y$?

Note that $\frac{x}{30}+\frac{y}{36}=\frac{6x+5y}{180}$.
Since $\left(6,\,5\right)=1$, there exists integer solution to equation
$6x+5y=1$, given that $1\divides 1$.

Therefore the smallest positive rational number that can be expressed
in the form of $\frac{x}{30}+ \frac{y}{36}$ is $\frac{1}{180}$.

% Question 12.
\subsubsection{12.} Let $m_1,\,\cdots,\,m_k$ be positive integers and $a,\,b\in\mathbb{Z}$.
Show that $a\equiv b$ (mod $m_i$) for each $i$ if and only if $a \equiv b$ (mod $\left[m_1,\,\cdots,\,m_k\right]$). % LCM

\begin{proof}
    By definition, if $a \equiv b$ (mod $m$), then $m \divides \left(a-b\right)$.
    Let $l$ = $\left[m_1,\,\cdots,\,m_k\right]$.

    ($\Rightarrow$) If $a \equiv b$ (mod $m_i$) for all $i$,
    then $m_i \divides \left(a-b\right)$ for all $i$.
    Suppose $l \ndivides \left(a-b\right)$. Then by the division algorithm, $\left(a-b\right) = lp + q$, where
    $p,\,q \in \mathbb{Z}$ and $1 \leq q < l$.

    Suppose $a\equiv b$ (mod $m_i$) for each $i$. Then $m_i \divides \left(a-b\right) = \left(lp + q\right)$ for all $i$.
    Since $m_i \divides l$, $m_i \divides q$ for all $i$, which means that $q$ is a common multiple of
    $m_1,\,m_2,\,\cdots,\,m_n$, but the fact that the LCM of $m_1,\,m_2,\,\cdots,\,m_n$ is $l$ and $1 \leq q < l$
    leads to contradiction. Thus if $a\equiv b$ (mod $m_i$) for each $i$, then $a \equiv b$ (mod $\left[m_1,\,\cdots,\,m_k\right]$).

    ($\Leftarrow$) If $a \equiv b$ (mod $l$), then $l \divides \left(a-b\right)$.
    Since $m_i \divides l$ for all $i$, $m_i \divides \left(a-b\right)$ for all $i$.
    Thus if $a \equiv b$ (mod $\left[m_1,\,\cdots,\,m_k\right]$), then $a\equiv b$ (mod $m_i$) for each $i$.
\end{proof}

\end{document}

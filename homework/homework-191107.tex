\documentclass[runningheads,11pt]{../llncs}
\usepackage{geometry}
\usepackage{graphicx}
\usepackage{wrapfig}
\usepackage{import}
\usepackage{kotex}
\usepackage[dvipsnames]{xcolor}
\usepackage{fancyvrb}
\usepackage{listings}
\usepackage{indentfirst}
\usepackage{tabularx}
\usepackage{underscore}
\usepackage{multicol}
\usepackage{enumitem}
\usepackage{menukeys}
\usepackage{amsmath}
\usepackage{clrscode3e} % https://www.ctan.org/pkg/clrscode3e?lang=en
\usepackage[numbers,square,super]{natbib}
\usepackage{inconsolata} % Inconsolata
\usepackage{mathptmx} % Times New Roman
\usepackage{amsmath}
\usepackage{amssymb}
\newcommand{\divides}{\bigm|}
\newcommand{\ndivides}{%
  \mathrel{\mkern.5mu % small adjustment
    % superimpose \nmid to \big|
    \ooalign{\hidewidth$\big|$\hidewidth\cr$\nmid$\cr}%
  }%
}
\setmainfont{Times New Roman}
\setmonofont{Fira Code}
\usepackage{minted}
\lstset{basicstyle=\footnotesize\ttfamily,breaklines=true}
\renewcommand{\bibname}{참고문헌}
\setlength{\parindent}{1em}
\setlength{\parskip}{1em}
\linespread{1.2}
{\renewcommand{\arraystretch}{1.5}%
\setlength{\tabcolsep}{0.5em}%
\newenvironment{Figure}
  {\par\medskip\noindent\minipage{\linewidth}}
  {\endminipage\par\medskip}
\newcommand{\translation}[1]{\textsuperscript{#1}}
\newcommand{\complexity}[1]{$\mathcal{O}\left({#1}\right)$}

\makeatletter
\renewcommand\NAT@citesuper[3]{\ifNAT@swa
\if*#2*\else#2\NAT@spacechar\fi
\unskip\kern\p@\textsuperscript{\NAT@@open#1\if*#3*\else,\NAT@spacechar#3\fi\NAT@@close}%
   \else #1\fi\endgroup}
\makeatother

\definecolor{codehighlight}{RGB}{0,118,191}
\definecolor{codebg}{RGB}{250,250,250}

\setminted{fontsize=\footnotesize, 
		   linenos,
		   numbersep=8pt,
		   gobble=0,
		   frame=lines,
		   bgcolor=codebg,
       framesep=3mm,
       highlightcolor=codehighlight!15,
       mathescape=true} 

\begin{document}

\title{MAT2120 Number Theory\newline Homework, November 7th}
\author{Suhyun Park (20181634)}
\institute{Department of Computer Science and Engineering, Sogang University}

\maketitle

\begin{lemma}[Gauss's Lemma]
    $p$ : odd prime. $p \ndivides a$. If $s$ is the number of least positive residues mod $p$ of the integers
    $a,\,2a,\,3a,\,\cdots,\,\frac{p-1}{2}a$ that are greater than $\frac{p}{2}$, then
    $\legendre{a}{p}=\left(-1\right)^s$.
\end{lemma}

\begin{proof}
    Let
    \begin{align*}
        z &= a \cdot 2a \cdot 3a \times \cdots \times \frac{p-1}{2}a \\ 
        &= a^{\frac{p-1}{2}} \left[1 \cdot 2 \cdot 3 \times \cdots \times \frac{p-1}{2}\right].
    \end{align*}
    
    Since $a$ and $p$ are coprime, $a,\,2a,\,\cdots,\,\frac{p-1}{2}a$ are distinct modulo $p$.

    If we define $f\left(x\right)$ to be
    \[
        f\left(x\right) = \begin{cases}
            x & \mbox{if } 1 \leq x \leq \frac{p-1}{2} \\
            p-x & \mbox{if } \frac{p+1}{2} \leq x \leq p-1
        \end{cases}
    \]
    then since $s$ is the count of least positive residues mod $p$ of the integers, it will count
    $\frac{p+1}{2} \leq ka \leq p-1$, hence
    \[
        z = \left(-1\right)^s \left[f\left(1\right) f\left(2\right) f\left(3\right) \times \cdots \times f\left(\frac{p-1}{2}\right)\right].
    \]

    Note that if, for some positive integer $1 \leq n,\,m \leq \frac{p-1}{2}$, $na \equiv \pm ma \pmod{p}$,
    then since $a$ is coprime to $p$, $n \equiv m \pmod{p}$ ($\because 1 \leq n,\,m \leq \frac{p-1}{2}$). This gives that
    $f\left(a\right),\,f\left(2a\right),\,\cdots,\,f\left(\frac{p-1}{2}a\right)$ is just a rearrangement of
    $1,\,2,\,\cdots,\,\frac{p-1}{2}$. Therefore since
    \[
        z=a^{\frac{p-1}{2}} \left[1 \cdot 2 \cdot 3 \times \cdots \times \frac{p-1}{2}\right]=\left(-1\right)^s \left[1 \cdot 2 \cdot 3 \times \cdots \times \frac{p-1}{2}\right],
    \]
    it is clear that $a^{\frac{p-1}{2}} = \left(-1\right)^s$. \qed
\end{proof}

\end{document}

\documentclass[runningheads,11pt]{../llncs}
\usepackage{geometry}
\usepackage{graphicx}
\usepackage{wrapfig}
\usepackage{import}
\usepackage{kotex}
\usepackage[dvipsnames]{xcolor}
\usepackage{fancyvrb}
\usepackage{listings}
\usepackage{indentfirst}
\usepackage{tabularx}
\usepackage{underscore}
\usepackage{multicol}
\usepackage{enumitem}
\usepackage{menukeys}
\usepackage{amsmath}
\usepackage{clrscode3e} % https://www.ctan.org/pkg/clrscode3e?lang=en
\usepackage[numbers,square,super]{natbib}
\usepackage{inconsolata} % Inconsolata
\usepackage{mathptmx} % Times New Roman
\usepackage{amsmath}
\usepackage{amssymb}
\newcommand{\divides}{\bigm|}
\newcommand{\ndivides}{%
  \mathrel{\mkern.5mu % small adjustment
    % superimpose \nmid to \big|
    \ooalign{\hidewidth$\big|$\hidewidth\cr$\nmid$\cr}%
  }%
}
\setmainfont{Times New Roman}
\setmonofont{Fira Code}
\usepackage{minted}
\lstset{basicstyle=\footnotesize\ttfamily,breaklines=true}
\renewcommand{\bibname}{참고문헌}
\setlength{\parindent}{1em}
\setlength{\parskip}{1em}
\linespread{1.2}
{\renewcommand{\arraystretch}{1.5}%
\setlength{\tabcolsep}{0.5em}%
\newenvironment{Figure}
  {\par\medskip\noindent\minipage{\linewidth}}
  {\endminipage\par\medskip}
\newcommand{\translation}[1]{\textsuperscript{#1}}
\newcommand{\complexity}[1]{$\mathcal{O}\left({#1}\right)$}

\makeatletter
\renewcommand\NAT@citesuper[3]{\ifNAT@swa
\if*#2*\else#2\NAT@spacechar\fi
\unskip\kern\p@\textsuperscript{\NAT@@open#1\if*#3*\else,\NAT@spacechar#3\fi\NAT@@close}%
   \else #1\fi\endgroup}
\makeatother

\definecolor{codehighlight}{RGB}{0,118,191}
\definecolor{codebg}{RGB}{250,250,250}

\setminted{fontsize=\footnotesize, 
		   linenos,
		   numbersep=8pt,
		   gobble=0,
		   frame=lines,
		   bgcolor=codebg,
       framesep=3mm,
       highlightcolor=codehighlight!15,
       mathescape=true} 

\begin{document}

\title{MAT2120 Number Theory\newline\space Problems II}
\author{Suhyun Park (20181634)}
\institute{Department of Computer Science and Engineering, Sogang University}

\maketitle

% Question 1.
\subsubsection{1.} Find all solutions of $87x \equiv 57$ (mod 105). 

\paragraph{Solution.}
\[
    87x \equiv 57 \mbox{ (mod 105)} \Leftrightarrow 29x \equiv 19 \mbox{ (mod 35)}.
\]
Using the extended Euclidean algorithm to find $a$, $b$ that satisfies
$29a + 35b = \left(29,\,35\right)= 1$ yields
\begin{align*}
    35 &= \underline{29} \cdot 1 + \underline{6} \\
    29 &= \underline{6} \cdot 4 + \underline{5} \\
    6 &= \underline{5} \cdot 1 + \underline{1} \\
    5 &= 1 \cdot 5 \\
    \therefore 1 &= 6 - 5 \cdot 1 \\
     &= 6 - \left(29 - 6 \times 4\right) \cdot 1 = 6 \times 5 - 29 \\
     &= \left(35 - 29 \cdot 1\right) \times 5 - 29 = 35 \times 5 - 29 \times 6 \\
     & \Rightarrow a = -6,\,b = 5.
\end{align*}
Hence,
\begin{align*}
    19\left(29a+35b\right)=19\times 1 &\Rightarrow 19a \times 29 = 19 - 35b \\
    &\Rightarrow 19a \times 29 \equiv 19 \mbox{ (mod 35)} \\
    & 29 \times \left(-114\right) \equiv 19 \mbox{ (mod 35)}.
\end{align*}
Thus $x \equiv -114 \equiv 26$ (mod 35),
so $x \equiv 26$ or $x \equiv 61$ or $x \equiv 96$ modulo 105.

% Question 2.
\subsubsection{2.} Show that $n\left(n-1\right)\left(2n-1\right)$ is
divisible by 6 for every positive integer $n$.

\begin{proof}
    Suppose $n=6k+r$ where $k \in \mathbb{Z}$ and $0 \leq r < 6$ and $r \in \mathbb{Z}$.
    Then
    \begin{align*}
        n\left(n-1\right)\left(2n-1\right) &= \left(6k+r\right)\left(6k+r-1\right)\left(12k+2r-1\right) \\
        &\equiv r\left(r-1\right)\left(2r-1\right) \\
        &\equiv 2r^3-3r^2+r \mbox{ (mod 6)}.
    \end{align*}
    \begin{enumerate}
        \item Suppose $r=2q_2+r_2$ where $q_2,\,r_2 \in \mathbb{Z}$ and $0\leq r_2<2$. Then
        \begin{align*}
            2r^3-3r^2+r &\equiv r^2 + r \\
            &\equiv \left(2q_2+r_2\right)^2 + \left(2q_2+r_2\right) \\
            &\equiv r_2^2 + r_2 \equiv r_2\left(r_2+1\right) \mbox{ (mod 2)}.
        \end{align*}
        Since $0\left(0+1\right) \equiv 1\left(1+1\right) \equiv 0$ (mod 2),
        $2 \divides n\left(n-1\right)\left(2n-1\right)$.
        \item Similarily, suppose $r=3q_3+r_3$ where $q_3,\,r_3 \in \mathbb{Z}$ and $0\leq r_3<3$. Then
        \begin{align*}
            2r^3-3r^2+r &\equiv 2r^3 + r \\
            &\equiv 2\left(3q_3+r_3\right)^3 + \left(3q_3+r_3\right) \\
            &\equiv 2r_3^3 + r_3 \equiv r_3\left(2r_3^2+1\right) \mbox{ (mod 3)}.
        \end{align*}
        Since $0\left(2\cdot 0^2+1\right) \equiv 1\left(2\cdot 1^2+1\right) \equiv 2\left(2\cdot 2^2+1\right) \equiv 0$ (mod 3),
        $3 \divides n\left(n-1\right)\left(2n-1\right)$.
    \end{enumerate}
    By 1. and 2., $2\times 3 = 6 \divides n\left(n-1\right)\left(2n-1\right)$ for any integer $n$.
\end{proof}

% Question 3.
\subsubsection{3.} What are the remainders when $3^{40}$ and $43^{37}$
are divided by 11?

\paragraph{Solution.}
\begin{enumerate}
    \item By Fermat's Little Theorem, $3^{11-1} \equiv 1$ (mod 11). Hence
    \begin{align*}
        3^{40} &\equiv \left(3^{10}\right)^4 \\
        &\equiv 1^4 \equiv 1 \mbox{ (mod 11)}.
    \end{align*}
    \item Note that $43 \equiv -1$ (mod 11). Thus
    \begin{align*}
        43^{37} &\equiv \left(-1\right)^{37} \\
        &\equiv -1 \equiv 10 \mbox{ (mod 11)}.
    \end{align*}
\end{enumerate}

% Question 4.
\subsubsection{4.} Find all solutions to the pair of congrugences
$3x-7y \equiv 4$ (mod 15), $7x-3y \equiv 1$ (mod 15).

\paragraph{Solution.} Since
\begin{align*}
    \left(3x-7y\right)\times 3 - \left(7x-3y\right)\times 7 &\equiv 4\times 3 - 1\times 7 = 5 \mbox{ (mod 15)} \\
    \Rightarrow 9x - 21x - 49x + 21y &\equiv 5 \mbox{ (mod 15)} \\
    \Rightarrow -40x \equiv 5x &\equiv 5 \mbox{ (mod 15)} \\
    \Rightarrow x &\equiv 1 \mbox{ (mod 3)} \\
    \Rightarrow x &\equiv 1 + 5k \mbox{ (mod 15)}
\end{align*}
where $k \in \mathbb{Z}$,
\begin{align*}
    3x-7y &\equiv 4 \mbox{ (mod 15)} \\
    \Rightarrow 3\left(1+5k\right) -7y &\equiv 4 \mbox{ (mod 15)} \\
    \Rightarrow 3+15k-7y &\equiv 4 \mbox{ (mod 15)} \\
    \Rightarrow 8y &\equiv 1 \mbox{ (mod 15)}.
\end{align*}
Thus $x \equiv 1$ (mod 3) and $y \equiv 2$ (mod 15).

% Question 5.
\subsubsection{5.} Find all integers between 3000 and 5000 that leave remainders of
1, 3, and 5 when divided by 7, 11, and 13, respectively.

\paragraph{Solution.} Let $x$ be an integer such that
\[
    \begin{cases}
        x \equiv 1 & \mbox{(mod 7)} \\
        x \equiv 3 & \mbox{(mod 11)} \\
        x \equiv 5 & \mbox{(mod 13)}
    \end{cases}.
\]
We derive $x$ by using the Chinese remainder theorem. Note that the solution of
\begin{align*}
    11\times 13 \times m_1 \equiv 3m_1 & \equiv 1 \mbox{ (mod 7)} \\
    7\times 13 \times m_2 \equiv 3m_2 & \equiv 3 \mbox{ (mod 11)} \\
    7\times 11 \times m_3 \equiv -m_3 & \equiv 5 \mbox{ (mod 13)}
\end{align*}
is given by $m_1 \equiv 5$ (mod 7), $m_2 \equiv 1$ (mod 11), $m_3 \equiv -5$ (mod 13),
thus
\[
    x \equiv 5\cdot 11\cdot 13 + 1 \cdot 7 \cdot 13 - 5 \cdot 7 \cdot 11 \equiv 421 \mbox{ (mod 1001)}.
\]
Hence integers between 3000 and 5000 that leave remainders of
1, 3, and 5 when divided by 7, 11, and 13 are:
\[
    3424,\, 4425.
\]

% Question 6.
\subsubsection{6.} Find the remainder when $13 \cdot 12^{45}$ is divided by 47.

\paragraph{Solution.} $13\cdot 12^{45}=12^{46}+12^{45}$. By Fermat's little theorem, 
$12^{47-1} \equiv 1$ (mod 47). Hence
\begin{align*}
    13\cdot 12^{45}=12^{46}+12^{45} &\equiv 1 + 12^{45} \\
    &\equiv 1+\left(12^2\right)^{22} \times 12 \equiv 1+\left(47 \times 3 + 3\right)^{22} \times 12 \\
    &\equiv 1+ 3^{22} \times 12 \\
    &\equiv 1+ \left(3^5\right)^4 \times 9 \times 12 \equiv 1+\left(47\times 5 + 8\right)^4 \times 108 \\
    &\equiv 1+8^4 \times \left(47 \times 2 + 14\right) \equiv 1 + 8^4 \times 14 \\
    &\equiv 1+\left(8^2\right)^2\times 14 \equiv 1+\left(47+17\right)^2\times 14 \\
    &\equiv 1+17^2\times 14 \\
    &\equiv 4047 \equiv 5.
\end{align*}

% Question 7.
\subsubsection{7.} Let $p$ and $q$ be distinct odd primes such that
$p-1$ divides $q-1$. If $\left(a,\,pq\right)=1$, prove that $a^{q-1}\equiv 1$ (mod $pq$).

\begin{proof}
    By Fermat's theorem, it is clear that $a^{q-1} \equiv 1$ (mod $q$) and
    $a^{p-1} \equiv 1$ (mod $p$).

    Since $\left(p-1\right)\divides \left(q-1\right)$, we can let $\left(q-1\right)=k\left(p-1\right)$,
    where $2 \leq k$ and $k \in \mathbb{Z}$. Then
    \begin{align*}
        \left(a^{p-1}\right)^k \equiv 1^k &\equiv 1 \mbox{ (mod $p$)} \\
        \Rightarrow a^{k\left(p-1\right)} &\equiv 1 \mbox{ (mod $p$)} \\
        \Rightarrow a^{q-1} &\equiv 1 \mbox{ (mod $p$)}.
    \end{align*}
    Since $p$ and $q$ are distinct primes; i. e. $\left(p,\,q\right)=1$, and since
    $a^{q-1}$ is congrugent to 1 both modulo $p$ and $q$, $a^{q-1}\equiv 1$ (mod $pq$).
    \qed
\end{proof}

% Question 8.
\subsubsection{8.} Show that if $a$ is not divisible by 2 or by 5, then $a^{101}$
ends in the same three decimal digits as does $a$. (Here we use the convention that
21, for example, ends with 021.)

\begin{proof}
    We want to show that $a^{101} \equiv a$ (mod 1000).
    
    Since $a$ is not divisible by 2 nor 5 but the only prime factor of 125 is 5,
    $\left(a,\,125\right)=1$. Note that
    \begin{align*}
        \phi\left(125\right) &= 125 \left(1-\frac{1}{5}\right) \\
        &= 100.
    \end{align*}
    By Euler's theorem, $a^{\phi\left(125\right)} = a^{100} \equiv 1$ (mod 125).

    Also for 8, since only prime factor of 8 is 2 and $\phi\left(8\right) = 4$,
    By Euler's theorem, $a^{\phi\left(8\right)} = a^{4} \equiv 1$ (mod 8). Hence,
    $\left(a^{4}\right)^{25} \equiv a^{100} \equiv 1$ (mod 8).

    Thus since $\left(8,\,125\right)=1$, $a^{100} \equiv 1$ (mod 1000); therefore $a^{101} \equiv a$ (mod 1000).
    \qed
\end{proof}

% Question 9.
\subsubsection{9.} Explain why every year has at least one Friday the 13\textsuperscript{th}.

\begin{proof}
    If January begins on day $k$, where $0 \leq k < 7$ and $k = 0$ being Sunday,
    then on a non-leap year,
    \begin{itemize}
        \item February begins on day $k+31 \equiv k+3$ (mod 7)
        \item March begins on day $k+3+28 \equiv k+3$ (mod 7)
        \item April begins on day $k+3+31 \equiv k+6$ (mod 7)
        \item May begins on day $k+6+30 \equiv k+1$ (mod 7)
        \item June begins on day $k+1+31 \equiv k+4$ (mod 7)
        \item July begins on day $k+4+30 \equiv k+6$ (mod 7)
        \item August begins on day $k+6+31 \equiv k+2$ (mod 7)
        \item September begins on day $k+2+31 \equiv k+5$ (mod 7)
        \item October begins on day $k+5+30 \equiv k$ (mod 7)
        \item November begins on day $k+31 \equiv k+3$ (mod 7)
        \item December begins on day $k+3+30 \equiv k+5$ (mod 7)
    \end{itemize}
    Then the set of starting days of each month forms a complete residue system, hence
    the set of days of the 13\textsuperscript{th} of each month also does. Similarily, this
    also holds in leap years. Thus it is guaranteed that every year will have at least one
    Friday the 13\textsuperscript{th}.
\end{proof}

\end{document}

\documentclass[runningheads,11pt]{../llncs}
\usepackage{geometry}
\usepackage{graphicx}
\usepackage{wrapfig}
\usepackage{import}
\usepackage{kotex}
\usepackage[dvipsnames]{xcolor}
\usepackage{fancyvrb}
\usepackage{listings}
\usepackage{indentfirst}
\usepackage{tabularx}
\usepackage{underscore}
\usepackage{multicol}
\usepackage{enumitem}
\usepackage{menukeys}
\usepackage{amsmath}
\usepackage{clrscode3e} % https://www.ctan.org/pkg/clrscode3e?lang=en
\usepackage[numbers,square,super]{natbib}
\usepackage{inconsolata} % Inconsolata
\usepackage{mathptmx} % Times New Roman
\usepackage{amsmath}
\usepackage{amssymb}
\newcommand{\divides}{\bigm|}
\newcommand{\ndivides}{%
  \mathrel{\mkern.5mu % small adjustment
    % superimpose \nmid to \big|
    \ooalign{\hidewidth$\big|$\hidewidth\cr$\nmid$\cr}%
  }%
}
\setmainfont{Times New Roman}
\setmonofont{Fira Code}
\usepackage{minted}
\lstset{basicstyle=\footnotesize\ttfamily,breaklines=true}
\renewcommand{\bibname}{참고문헌}
\setlength{\parindent}{1em}
\setlength{\parskip}{1em}
\linespread{1.2}
{\renewcommand{\arraystretch}{1.5}%
\setlength{\tabcolsep}{0.5em}%
\newenvironment{Figure}
  {\par\medskip\noindent\minipage{\linewidth}}
  {\endminipage\par\medskip}
\newcommand{\translation}[1]{\textsuperscript{#1}}
\newcommand{\complexity}[1]{$\mathcal{O}\left({#1}\right)$}

\makeatletter
\renewcommand\NAT@citesuper[3]{\ifNAT@swa
\if*#2*\else#2\NAT@spacechar\fi
\unskip\kern\p@\textsuperscript{\NAT@@open#1\if*#3*\else,\NAT@spacechar#3\fi\NAT@@close}%
   \else #1\fi\endgroup}
\makeatother

\definecolor{codehighlight}{RGB}{0,118,191}
\definecolor{codebg}{RGB}{250,250,250}

\setminted{fontsize=\footnotesize, 
		   linenos,
		   numbersep=8pt,
		   gobble=0,
		   frame=lines,
		   bgcolor=codebg,
       framesep=3mm,
       highlightcolor=codehighlight!15,
       mathescape=true} 

\begin{document}

\title{MAT2120 Number Theory\newline Homework, September 26th}
\author{Suhyun Park (20181634)}
\institute{Department of Computer Science and Engineering, Sogang University}

\maketitle

\begin{theorem}
    For $a>0,\,b>0\in\mathbb{Z}$,
    \[
        \left(a,\,b\right)\left[a,\,b\right] = ab.    
    \]
\end{theorem}

\begin{lemma}\label{lemma1}
    For $k > 0 \in \mathbb{Z}$, $\left[ka,\,kb\right] = k\left[a,\,b\right]$.
\end{lemma}
\begin{proof}
    Since $ka \divides \left[ka,\,kb\right]$, $k \divides \left[ka,\,kb\right]$.

    Let $l = \left[a,\,b\right]$ and $km = \left[ka,\,kb\right]$.
    Then
    \[
        ka \divides km \Rightarrow a \divides m \qquad \mbox{and} \qquad
        kb \divides km \Rightarrow b \divides m.
    \]
    Hence $m$ is a common multiple of $a$ and $b$, giving that
    \begin{equation}
        m \geq l \label{eqn1}
    \end{equation}
    because $l$ is the least common multiple of $a$ and $b$.

    Similarily, 
    \[
        a \divides l \Rightarrow ka \divides kl \qquad \mbox{and} \qquad
        b \divides l \Rightarrow kb \divides kl.
    \]
    Hence $kl$ is a common multiple of $ka$ and $kb$, giving that
    \begin{equation}
        kl \geq km \Rightarrow l \geq m \label{eqn2}
    \end{equation}
    because $km$ is the least common multiple of $ka$ and $kb$.

    By \eqref{eqn1} and \eqref{eqn2}, we can conclude that $l=m$. Therefore
    \[
        \left[ka,\,kb\right] = km = kl = k\left[a,\,b\right].
    \]
    \qed
\end{proof}

\begin{proof}[Conclusion of Proof of Theorem]
    Let $d=\left(a,\,b\right)$. Then $d \divides a$ and $d \divides b$ is true by definition.

    Hence we let $a=a_0d$, $b=b_0d$. Then $\left(a_0,\,b_0\right)=1$.

    Now we want to show that $\left[a_0,\,b_0\right]=a_0b_0$. Since $a_0 \divides \left[a_0,\,b_0\right]$,
    we let $\left[a_0,\,b_0\right]=ka_0$.

    Since $b_0 \divides \left[a_0,\,b_0\right] \Rightarrow b_0 \divides ka_0$ and $\left(a_0,\,b_0\right)=1$,
    we know that $b_0 \divides k$. Thus, $b_0a_0 \leq ka_0$.

    Note that $ka_0$ is the least common multiple of $a_0$, $b_0$ and
    $b_0a_0$ is the common multiple of $a_0$, $b_0$, thus $b_0a_0 \geq ka_0$.
    Hence $a_0b_0 = ka_0 = \left[a_0,\,b_0\right]$.

    Using Lemma \ref{lemma1}, we can conclude that
    \begin{align*}
        & \left(a,\,b\right)\left[a,\,b\right] \\  
        =& d \left[a_0d,\,b_0d\right] \\
        =& d^2 \left[a_0,\,b_0\right] \\
        =& d^2a_0b_0 = \left(da_0\right)\left(db_0\right) \\
        =& ab.
    \end{align*}
    \qed
\end{proof}

\begin{theorem}
    Let $b$ be a positive integer with $b>1$. Then every positive integer
    $n$ can be expressed in unique form of
    \[
        n=a_kb^k + a_{k-1}b^{k-1} + \cdots + a_1b^1 + a_0
    \]  
    where $a_i \in \mathbb{Z}$, $0 \leq a_i \leq b-1$ for $i=0,\,1,\,\cdots,\,k$
    and $a_k \neq 0$.
\end{theorem}

\begin{proof}
    We use the division algorithm. For $0\leq a_0 < b$, we can express $n$ as
    \[
        n = q_0 b + a_0 .
    \]
    If $q_0 \geq b$, we can express $q_0$ as
    \[
        q_0 = q_1 b + a_1 .    
    \]
    We can repeat this process for $q_i$ while $q_i \geq b$. This gives
    \begin{align*}
        n &= q_0 b^1 + a_0 \\
        &= q_1 b^2 + a_1 b^1 + a_0 \\
        & \vdots \\
        &= a_kb^k + a_{k-1}b^{k-1} + \cdots + a_1b^1 + a_0. 
    \end{align*}

    Now we want to show that such $a_0,\, a_1,\, \cdots,\, a_k$ uniquely exists.
    Let
    \[
        n=a^\prime_kb^k + a^\prime_{k-1}b^{k-1} + \cdots + a^\prime_1b^1 + a^\prime_0.
    \]
    Then
    \begin{align*}
        a_kb^k + a_{k-1}b^{k-1} + \cdots + a_1b^1 + a_0&=a^\prime_kb^k + a^\prime_{k-1}b^{k-1} + \cdots + a^\prime_1b^1 + a^\prime_0\\
        \Rightarrow q_0b^1 + a_0&=q^\prime_0b^1 + a^\prime_0.\\
        \Rightarrow \left(q_0 - q^\prime_0\right)b^1 &= a^\prime_0 - a_0.\\
    \end{align*}
    If $q_0 \neq q^\prime_0$, $b \divides \left(a^\prime_0 - a_0\right)$, but since $0\leq a_0, a^\prime_0 < b$, $-b<a^\prime_0 - a_0<b$, 
    which falls into contradiction. Hence $q_0 = q^\prime_0$ and also $a_0 = a^\prime_0$. Similarlily we can repeat this process for
    $q_i$ to show that $a_i = a^\prime_i$ for $0 \leq i \leq k$, proving that $a_0,\, a_1,\, \cdots,\, a_k$ uniquely exists. \qed
\end{proof}

\end{document}

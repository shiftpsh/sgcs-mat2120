\documentclass[runningheads,11pt]{llncs}
\usepackage{geometry}
\usepackage{graphicx}
\usepackage{wrapfig}
\usepackage{import}
\usepackage{kotex}
\usepackage[dvipsnames]{xcolor}
\usepackage{listings}
\usepackage{tabularx}
\usepackage{underscore}
\usepackage{multicol}
\usepackage{enumitem}
\usepackage[numbers,square,super]{natbib}
\usepackage{mathptmx} % Times New Roman
\usepackage{amsmath}
\usepackage{amssymb}
\usepackage{framed}
\usepackage{etoolbox}
\usepackage{cancel}

\colorlet{shadecolor}{gray!30}

\newcommand\enclosebox[2]{%
  \BeforeBeginEnvironment{#1}{\begin{#2}}%
  \AfterEndEnvironment{#1}{\end{#2}}%
}

\enclosebox{theorem}{oframed}
\enclosebox{definition}{leftbar}

\newcommand{\divides}{\bigm|}
\newcommand{\ndivides}{%
  \mathrel{\mkern.5mu % small adjustment
    % superimpose \nmid to \big|
    \ooalign{\hidewidth$\big|$\hidewidth\cr$\nmid$\cr}%
  }%
}
\newcommand{\ord}{\operatorname{\mathrm{ord}}}
\newcommand{\ind}{\operatorname{\mathrm{ind}}}
\setmainfont{Times New Roman}
\setmonofont{Fira Code}
\setlength{\parindent}{1em}
\setlength{\parskip}{1em}
\linespread{1.2}
{\renewcommand{\arraystretch}{1.5}%
\setlength{\tabcolsep}{0.5em}%
\newenvironment{Figure}
  {\par\medskip\noindent\minipage{\linewidth}}
  {\endminipage\par\medskip}
\newcommand{\translation}[1]{\textsuperscript{#1}}

\makeatletter
\renewcommand\NAT@citesuper[3]{\ifNAT@swa
\if*#2*\else#2\NAT@spacechar\fi
\unskip\kern\p@\textsuperscript{\NAT@@open#1\if*#3*\else,\NAT@spacechar#3\fi\NAT@@close}%
   \else #1\fi\endgroup}
\makeatother
\usepackage{pgfplots}
\pgfplotsset{compat=newest}

\begin{document}

\title{MAT2120 Number Theory}
\author{Suhyun Park (20181634)}
\institute{Department of Computer Science and Engineering, Sogang University}

\subsection{September 19th}

\begin{proof}
    $S \subset \mathbb{Z} \Rightarrow S = m\mathbb{Z}$

    Since $S \neq \emptyset$, $\exists a \in S$

    Since $S$ is closed under $+$, $-$, $0 \in S$. We may assume that
    $S \neq \left\{0\right\}$. (if $S = \left\{0\right\}$, then $S = 0\cdot \mathbb{Z}$)

    Take any $n \in S$. Then $0 - n = -n \in S$.
    Thus we may also assume that $S$ has a positive integer.

    In all, WLOG\footnote{Without loss of generality},
    we may assume that $S$ has a positive integer.

    By WOP, $S$ has a least positive integer $m$. We want to show that $S=m\mathbb{Z}$.

    \marginpar{
        $A = B \Rightarrow A \subset B \mbox{ and } B \subset A$
        
        $A \subset B \Rightarrow \mbox{if } x \in A \mbox{ then } x \in B$
    }

    \begin{enumerate}
        \item $m\mathbb{Z} \subset S$
        
        $m \in S$ and $S$ is closed under $+$, $-$. So $S$ must have all multiples of $m$.

        \item $S \subset m\mathbb{Z}$
        
        Take any $a \in S$. By division algorithm, $\exists q,\, r \in \mathbb{Z}$ such that
        $a = qm + r$ where $0 \leq r < m$.
        Since $mq \in S$ and $a \in S$, \[r = a - mq \in S\].
        Thus $r = 0$ by the minimality of $m$. Hence $a = mq \in m\mathbb{Z}$.

        Remains to show the uniqueness of $m$. Suppose $m\mathbb{Z} = S = m^\prime\mathbb{Z}$.
        Then $m = \pm m^\prime$.
        Since $m,\,m^\prime > 0$, $m=m^\prime$.
    \end{enumerate}
\end{proof}

\begin{theorem}
    Let $d=\left(a,\,b\right)$. Then $d=ax+by$ for some $x,\,y \in \mathbb{Z}$
    and $\left\{ax+by \mid x,\,y \in \mathbb{Z}\right\}$ is the set of all multiples of $d$.
    i. e. $a\mathbb{Z}+b\mathbb{Z} = \left\{ax+by \mid x,\,y \in \mathbb{Z}\right\}$.
\end{theorem}

\begin{proof}
    We knew that $d=ax+by$ for some $x,\,y \in \mathbb{Z}$. (by the theorem in the last class)

    Define $S := a\mathbb{Z} + b\mathbb{Z}$. Then $a\mathbb{Z} \subset S$ and $b\mathbb{Z} \subset S$.
    Since $S$ is closed under $+$, $-$, it follows the previous theorem that
    \[\exists m \geq 0 \in \mathbb{Z} \mbox{ such that } S = m\mathbb{Z}.\]

    We want to show that $m=d$. Since $a,\,b \in S = m\mathbb{Z}$, $m \divides a$, $m \divides b$.
    If $e \divides a$ and $e \divides b$, then $e \divides m$.
    ($\because m + as + bt \mbox{ for some } s,\,t \in \mathbb{Z}$)
    
    By the definition of GCD, $m=d$.
\end{proof}

\paragraph{Remark}
The GCD of $a$ and $b$ (not both 0)
is the least positive integer that is a linear combination of $a$ and $b$.

\begin{theorem}[Euclidean Algorithm]
    $a,\,b \in \mathbb{Z}$, $a \neq 0$.
    Using the division algorithm, \[b = aq_1 + r_1\mbox{, where }0<r_1<\left|a\right|.\]
    If $r_1 = 0$, terminate process.

    Repeating process,
    \begin{align*}
        a &= r_1q_2 + r_2 & 0<r_2<r_1 \\
        r_1 &= r_2q_3 + r_3 & 0<r_3<r_2 \\
        &\vdots \\
        r_{n-2} &= r_{n-1}q_n + r_n & 0<r_n<r_{n-1} \\
        r_{n-1} &= r_nq_{n+1}
    \end{align*}
    Then $\left(a,\,b\right)=r_n$.
\end{theorem}

\begin{proof}
    Clearly, $r_n > 0$.
    Note that
    \begin{align*}
        r_n \divides r_{n-1}, r_n \divides r_n &\Rightarrow r_n \divides r_{n-2} \\
        r_n \divides r_{n-2}, r_n \divides r_{n-1} &\Rightarrow r_n \divides r_{n-3} \\
        &\vdots \\
        r_n \divides r_1, r_n \divides r_2 &\Rightarrow r_n \divides a \\
        r_n \divides a, r_n \divides r_1 &\Rightarrow r_n \divides b
    \end{align*}

    Note also that if
    \begin{align*}
        k \divides a,\, k \divides b &\Rightarrow k\divides r_1 \\
        k \divides r_1,\, k \divides a &\Rightarrow k\divides r_2 \\
        &\vdots \\
        k \divides r_n,\, k \divides r_{n-1} &\Rightarrow k\divides r_n
    \end{align*}
    
    Hence we conclude that $r_n = \left(a,\,b\right)$.
\end{proof}

\begin{proof}[Alternate proof]
    \[
        b = aq + r \Rightarrow \left(a,\,b\right) = \left(a,\,r\right)
        \qquad
        r=a\left(-q\right)+b,\,b=aq+r
    \]
    Note that $e\divides a$, $e\divides b$ iff $e\divides r$, $e\divides a$.
    Thus $\left(a,\,b\right)\divides\left(a,\,b\right)$ and 
    $\left(a,\,k\right)\divides\left(a,\,b\right)$.

    Hence $\left(a,\,b\right) = \left(a,\,r\right)$, since $\left(a,\,b\right) > 0$ and
    $\left(a,\,k\right) > 0$. Therefore we can see that
    \[
        \left(a,\,b\right) = \left(a,\,r\right) = \left(r_1,\,r_2\right) = \cdots
        = \left(r_{n-1},\,r_n\right).
    \]
\end{proof}

\paragraph{Example}
\begin{align*}
    \left(68,\,710\right) &= 2 \\
    710 &= 68 \cdot 10 + 30 \\
    68 &= 30 \cdot 2 + 8 \\
    30 &= 8 \cdot 3 + 6 \\
    8 &= 6 \cdot 1 + 2 \\
    6 &= 2 \cdot 3
\end{align*}
\begin{align*}
    2 &= 8 - 6\cdot 1 \\
    &= 8 - \left(30 - 8 \cdot 3\right) \\
    &= 8 \cdot 4 + 30 \cdot \left(-1\right) \\
    &= \left(68-30\cdot 2\right) \cdot 4 + 30\cdot\left(-1\right) \\
    &= 68 \cdot 4 + 30 \cdot \left(-1\right) \\
    &= 68 \cdot 4 + \left(710 - 68 \cdot 10\right) \cdot \left(-9\right) \\
    &= 68 \cdot 94 + 710 \cdot \left(-9\right)
\end{align*}

\begin{definition}[Diophantine Equation]
    A \textbf{Diophantine equation} is a polynomial equation that allows two or more variables
    to take integer values only.
\end{definition}

e. g.
\[ax+by=c\] \[x^n+y^n=z^n\] \[x^2-dy^2=1\]

\begin{theorem}
    $a\neq 0$, $b\neq 0$.
    \begin{enumerate}
        \item The equation $ax+by=c$ has integer solutions if and only if
        $\left(a,\,b\right) \divides c$.
        \item Suppose that $\left(a,\,b\right) \divides c$.
        Then the general solution of the equation $ax+by=c$ has form the of
        \[
            \left\{x_0 + \frac{b}{\left(a,\,b\right)}t,\, y_0 - \frac{a}{\left(a,\,b\right)}t\right\}    
        \]
        where $t \in \mathbb{Z}$ and $\left(x_0,\,y_0\right)$ is an arbitrary solution of the equation.
    \end{enumerate}
\end{theorem}

\marginpar{
    General solution for $y^{\prime\prime}-4y^\prime+3y = 0$?

    $\Rightarrow c_1e^x + c_2e^{3x}$
    
    -- 2 bases
}

\subsection{September 24th}

\begin{proof}
    Note that

    \begin{align*}
        & a \divides b,\, a \divides c \Rightarrow a \divides \left(bx+cy\right) \qquad\forall x,\,y\in\mathbb{Z} \\
        & m \divides ab,\, \left(m,\,a\right)=1 \Rightarrow m\divides b \qquad \because \left(m,\,a\right)=1,\, \exists s,\,t \in \mathbb{Z} \qquad as+mt=1
    \end{align*}

    Then $bas+bmt=b$.

    Since $m\divides ab$, it follows that $m \divides b$.

    \begin{enumerate}
        \item ($\Rightarrow$)
        $\left(a,\,b\right)\divides a$, $\left(a,\,b\right)\divides b$ $\Rightarrow$
        $\left(a,\,b\right)\divides \left(ax+by\right)=c$

        ($\Leftarrow$)
        Let $\left(a,\,b\right)=d$ and $c=c_1d$. Then $\exists s,\,t \in \mathbb{Z}$ such that $as+bt = d$.
        thus
        \begin{align*}
            c=c_1d &= c_1\left(as+bt\right) \\
            &=ac_1s + bc_1t
        \end{align*}
        hence $\left(c_1s,\,c_1t\right)$ is a solution.

        \item Note that
        \begin{align*}
            &a\left(x_0+\frac{b}{d}t\right) + b\left(y_0-\frac{a}{d}t\right)\\
            =&ax_0+\frac{ab}{d}t+by_0-\frac{ba}{d}t \\
            =& ax_0+by_0=c
        \end{align*}
        
        Suppose that $\left(x,\,y\right)$ is an arbitrary solution of $ax+by=c$.
        Since $ax+by=c=ax_0+by_0$, we have
        \[
            a\left(x-x_0\right)=b\left(y_0-y\right).
        \]
        Let $a=a_1d$, $b=b_1d$, where $d=\left(a,\,b\right)$. Then
        \[
            a_1\left(x-x_0\right)=b_1\left(y_0-y\right).
        \]
        Since $\left(a,\,b\right)=1$, $b_1\divides \left(x-x_0\right)$. Then
        $\exists t \in \mathbb{Z}$ such that $x-x_0=b_1t$, and similarily $y_0-y=a_1t$.

        Hence \[
            x=x_0+\frac{b}{\left(a,\,b\right)}t,\, y=y_0-\frac{a}{\left(a,\,b\right)}t
        .\]
    \end{enumerate}
\end{proof}

\paragraph{Example}
\[
    710 x + 68 y = 6
\]\footnote{Maybe an eaxm problem?}
Recall
\begin{align*}
    710\cdot\left(-9\right)+68\cdot94&=2 \\
    710\cdot\left(-9\times 3\right)+68\cdot\left(94\times 3\right)&=2\times3=6 \\
\end{align*}
Hence
\begin{align*}
    x&=-27+\frac{68}{2}t=-27+34t \\
    x&=282-\frac{710}{2}t=282-355t
\end{align*}

\begin{definition}[Least Common Multiple]
    The \textbf{least common multiple} of two nonzero integers $a$ and $b$,
    denoted $\left[a,\,b\right]$ or $\mathop{\mathrm{lcm}}\left(a,\,b\right)$
    is the integer $l$ satisfying the followings:
    \begin{enumerate}
        \item $l>0$.
        \item $a\divides l$, $b\divides l$.
        \item $a\divides c$, $b\divides c$ $\Rightarrow m\divides c$.
    \end{enumerate}
\end{definition}

\begin{theorem}
    For $a\neq 0,\,b\neq 0 \in \mathbb{Z},$ $\left[a,\,b\right]$ uniquely exists.
    Moreover, $a\mathbb{Z} \cap b\mathbb{Z} = \left[a,\,b\right]\mathbb{Z}$.
\end{theorem}

\begin{proof}
    Let $S=a\mathbb{Z} \cap b\mathbb{Z}$.
    Since $ab \in S$, $S \neq \emptyset$. Clearly, $S$ is closed under $+$, $-$.

    By theorem, $\exists l$ such that $S = l\mathbb{Z}$.

    We want to show that $l=\left[a,\,b\right]$. Since $l\in S$, $a\divides l$,
    $b\divides l$. If $a\divides c$, $b\divides c$, then $c \in S = l\mathbb{Z}$ and $l\divides c$.

    Remains to show the uniqueness of $l$. Suppose $l_1$ and $l_2$ are both the
    LCMs of $a$ and $b$. Then $l_1 \divides l_2$ and $l_2 \divides l_1$.
    By (2), (3), $l_1=l_2$, since $l_1>0,\,l_2>0$.
\end{proof}

\paragraph{Remark}

\begin{align*}
    \left(a,\,b\right)\mathbb{Z} &= a\mathbb{Z} + b\mathbb{Z} = \left\{ax+by \mid x,\,y\in\mathbb{Z}\right\} \\
\end{align*}

\paragraph{Recall}
\begin{enumerate}
    \item $\left(0,\,0\right) := 0$
    \item $\left(a,\,0\right) := \left|a\right|$
    \item $\left[0,\,0\right] := 0$
    \item $\left[a,\,0\right] := 0$
\end{enumerate}

\begin{theorem}
    For $a>0,\,b>0\in\mathbb{Z}$, \[\left(a,\,b\right)\left[a,\,b\right] = ab\].
\end{theorem}

\begin{proof}
    (Proof left for homework -- due September 26th.)
\end{proof}

\begin{theorem}
    Let $b$ be a positive integer with $b>1$. Then every positive integer
    $n$ can be expressed in unique form of
    \[
        n=a_kb^k + a_{k-1}b^{k-1} + \cdots + a_1b^1 + a_0
    \]  
    where $a_i \in \mathbb{Z}$, $0 \leq a_i \leq b-1$ for $i=0,\,1,\,\cdots,\,k$
    and $a_k \neq 0$.
\end{theorem}
\marginpar{$b$ $\Rightarrow$ base.}

\begin{proof}
    We use the division algorithm. (Proof left for homework -- due September 26th.)
\end{proof}

\begin{definition}[Prime Numbers]
    A \textbf{prime} is an integer $p$ such that
    \begin{enumerate}
        \item $p>1$.
        \item $a\divides p \Rightarrow a = \pm 1 \mbox{ or } \pm p$.
    \end{enumerate}
\end{definition}

\paragraph{Remark}
$p$ is prime.

\begin{enumerate}
    \item $\forall a \in \mathbb{Z},\, \left(a,\,p\right) = 1 \mbox{ or } \left(a,\,p\right) = p$. (iff $p$ is prime)
    \item $p \divides ab \Rightarrow p \divides a \mbox{ or } p \divides b$. (iff $p$ is prime)
\end{enumerate}

\begin{theorem}[Infinitude of Primes]
    There exists infinitely many primes.
\end{theorem}

\begin{proof}[Euclid's]
    \begin{lemma}
        Every positive integer $n \geq 2$ has a prime factor.
    \end{lemma}

    \begin{proof}
        Consider the set $S=\left\{m \mid m\mbox{ is a divisor of }n\right\}$.
        Then $S \neq \emptyset$.

        By WOP, $\exists$ least positive integer $p\in S$. Note that
        every divisor of $p$ is also a divisor of $n$.
        Thus $p$ is a prime number by the minimality of $p$.
    \end{proof}

    Suppose there exists finitely many primes
    \[
        p_1,\,p_2,\,\cdots,\,p_k.    
    \]
    Let
    \[
        n := p_1p_2\times\cdots\times p_k.    
    \]
    Then $n > 1$ and $\exists$ prime $p$ such that $p\divides n$ by Lemma.

    Thus $p=p_i$ for some $1 \leq i \leq k$, hence $p \divides p_1p_2\times\cdots\times p_k$, thus
    \[
        p \divides \left(n-p_1p_2\times\cdots\times p_k\right)
        \Rightarrow p\divides 1.
    \]
    Which is a contradiction to the definition of prime numbers. Thus there exists infinitely many primes.
\end{proof}

\begin{theorem}
    There are arbitrary large gaps between successive primes.
    i. e. For any positive integer $n$, there exists at least $n$ consecutive composite
    positive integers.
\end{theorem}

\begin{proof}
    Consider $n$ consecutive integers
    \[
        \left(n+1\right)!+2,\,\left(n+1\right)!+3,\,\cdots,\,\left(n+1\right)!+\left(n+1\right).
    \]
    For $2\leq j \leq n+1$, it is clear that $j \divides \left(n+1\right)!$.
    Thus $j \divides \left(\left(n+1\right)!+j\right)$.

    Hence $\exists n$ consecutive integers which are all composites.
\end{proof}

\begin{definition}[Mersenne Primes]
    A \textbf{Mersenne prime} is a Mersenne number\footnote{$M_n = 2^n - 1$}
    which is also prime.
\end{definition}

e. g. $M_2 = 2^2-1 = 3$, $M_3 = 2^3-1 = 7$, $M_5 = 2^5-1 = 31$,
$M_7 = 2^7-1 = 127$, $\cdots$ but $M_{11} = 2^{11}-1 = 2047 = 23 \times 89$

\end{document}

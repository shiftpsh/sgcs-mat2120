\begin{proof}
    \begin{align*}
        k&=0 & C_0 &= \left[a_0\right] = \frac{a_0}{1} = \frac{a_0}{q_0} \\
        k&=1 & C_1 &= \left[a_0;,\,a_1\right] = a_0 + \frac{1}{a_1} = \frac{a_0a_1+1}{a_1} = \frac{p_1}{q_1} \\
        k&=2 & C_2 &= \left[a_0;,\,a_1,\,a_2\right] = a_0 + \frac{1}{a_1+\frac{1}{a_2}} = \frac{a_0\left(a_1a_2+1\right)+a_2}{a_1a_2+1} = \frac{p_2}{q_2}
    \end{align*}
    Assume that the results hold for a positive integer $2 \leq k \leq n$.
    Note that
    \begin{align*}
        C_{k+1} &= \left[a_0;\,a_1,\,a_2,\,\cdots,\,a_k\right] \\
        &= \left[a_0;\,a_1,\,a_2,\,\cdots,\,a_k+\frac{1}{a_{k+1}}\right] \\
        &= \frac{\left(a_k + \frac{1}{a_{k+1}}\right)p_{k-1}+p_{k-2}}{\left(a_k + \frac{1}{a_{k+1}}\right)q_{k-1}+q_{k-2}} \\
        &= \vdots \leftarrow \mbox{(see the text.)} \\
        &= \frac{p_{k+1}}{q_{k+1}}
    \end{align*}
\end{proof}

\begin{theorem}
    $k \in \mathbb{Z}^+$. $p_k q_{k-1} - p_{k-1} q_k = \left(-1\right)^{k-1}$.
\end{theorem}

\begin{proof}
    \begin{align*}
        k&=1 & p_1q_0 - p_0q_1 &= \left(a_0a_1+1\right)-a_0a_1 = 1 \\
        k&=2 & p_2q_1 - p_1q_2 &= \left(a_2p_1+a_0\right)a_1-p_1\left(a_2a_1+1\right) = -1
    \end{align*}
    Use induction for $k \geq 3$. (See the text)
\end{proof}

\begin{corollary}
    $p_k,\,q_k$ are relatively prime.
\end{corollary}

\begin{proof}
    Let $d_k = \left(p_k,\,q_k\right)$. By theorem,
    \[
        p_kq_{k-1}-p_{k-1}q_k = \left(-1\right)^{k-1}.
    \]
    Thus $d_k \divides \left(-1\right)^{k-1}$, hence $d_k = 1$.
\end{proof}

\begin{corollary}
    \begin{align*}
        C_k-C_{k-1} &= \frac{\left(-1\right)^{k-1}}{q_k q_{k-1}}&\mbox{ for }1 \leq k \leq n \\    
        C_k-C_{k-2} &= \frac{a_k\left(-1\right)^{k}}{q_k q_{k-2}}&\mbox{ for }2 \leq k \leq n        
    \end{align*}
\end{corollary}
    
\begin{proof}
    By theorem,
    \[
        p_kq_{k-1}-p_{k-1}q_k = \left(-1\right)^{k-1}.
    \]
    Then
    \begin{align*}
        C_k-C_{k-1} &= \frac{p_k}{q_k} - \frac{p_{k-1}}{q_{k-1}} \\
        &= \frac{p_kq_{k-1} - q_kp_{k-1}}{q_kq_{k-1}} \\
        &= \frac{\left(-1\right)^{k-1}}{q_kq_{k-1}}\\
        C_k-C_{k-2} &= \frac{p_k}{q_k} - \frac{p_{k-2}}{q_{k-2}} \\
        &= \frac{p_kq_{k-2} - q_kp_{k-2}}{q_kq_{k-2}} \\
        &= \frac{\left(a_kp_{k-1}+p_{k-2}\right)q_{k-2} - \left(a_kq_{k-1}+q_{k-2}\right)p_{k-2}}{q_kq_{k-2}} \\
        &= \frac{a_k\left(-1\right)^{k-2}}{q_kq_{k-2}} \\
        &= \frac{a_k\left(-1\right)^k}{q_kq_{k-2}}
    \end{align*}
\end{proof}

\begin{theorem}
    Let $C_k$ be the $k$th convergent of finite simple continued fraction
    $\left[a_0;\,a_1,\,a_2,\,\cdots,\,a_n\right]$. Then
    \[
        C_1>C_3>C_5>\cdots
    \]
    \[
        C_2<C_4<C_6<\cdots
    \]
    and
    \[
        C_{2j+1} > C_{2k} \quad\mbox{ for } j=0,\,1,\,2,\,\cdots \quad k=0,\,1,\,2,\,\cdots
    \]
    i. e. 
    \[
        C_0 < C_2 < C_4 < \cdots < C_5 < C_3 < C_1.
    \]
\end{theorem}

\begin{proof}
    By corollary,
    \[
        C_k-C_{k-2} = \frac{a_k \left(-1\right)^k}{q_k q_{k-2}}.
    \]
    Thus\[
        \begin{cases}
            C_k-C_{k-2} < 0 & \mbox{if $k$ is odd} \\
            C_k-C_{k-2} > 0 & \mbox{if $k$ is even} \\
        \end{cases}
    \]
    and so
    \[
        C_1>C_3>C_5>\cdots
    \]
    \[
        C_2<C_4<C_6<\cdots.
    \]
    Note that
    \[
        C_{2m} - C_{2m - 1} = \frac{\left(-1\right)^{2m-1}}{q_{2m}q_{2m-1}}<0,
    \]
    and so $C_{2m} < C_{2m-1}$. Therefore
    \[
        C_{2j+1} > C_{2j+2k+1} > C_{2j+2k+2} > C_{2k}
    \]
    for $j=0,\,1,\,2,\,\cdots$, $k=0,\,1,\,2,\,\cdots$.
\end{proof}

\begin{remark}
    Consider $79x \equiv 3 \pmod{103} \Leftrightarrow 79x + 103y = 3$.
    Note that
    \[
        \frac{79}{103}=\left[0;\,1,\,3,\,3,\,2,\,3\right].
    \]
    It can be seen that $p_5q_4-p_4q_5 = 79\cdot 30 - \left(-23 \right)\cdot 104 = 1$.
    Thus
    \[
        79x + 103y = 3 \Leftarrow \begin{cases}
            x = 90 \\ 
            y = -69
        \end{cases}
    \]
\end{remark}

Suppose that we have an infinite sequence of $a_0,\,a_1,\,a_2,\,\cdots$ such that
$a_0,\,a_1,\,\cdots > 0$. We define \textbf{infinite continued fractions} as limits
of finite continued fractions.    

\begin{theorem}
    Let $a_0,\,a_1,\,\cdots$ be an infinite sequence of integers with
    $a_0,\,a_1,\,\cdots > 0$, and let $C_k := \left[a_0;\,a_1,\,a_2,\,\cdots,\,a_k\right]$.
    Then $\lim_{k\rightarrow \infty} C_k = \alpha$ exists.
\end{theorem}

\begin{proof}
    Recall
    \[
        C_0 < C_2 < C_4 < \cdots < C_5 < C_3 < C_1.
    \]
    By Monotone Convergence Theorem, 
    \[
        \lim_{n \rightarrow \infty} C_{2n+1}=\alpha_1
        \qquad
        \lim_{n \rightarrow \infty} C_{2n}=\alpha_2
    \]
    for some $\alpha_1,\,\alpha_2 \in \mathbb{R}$. Note that
    \[
        C_{2n+1}-C_{2n} = \frac{\left(-1\right)^{2n}}{q_{2k+1}q_{2k}} = \frac{1}{q_{2k+1}q_{2k}}
    \]
    and $q_k>k$ $\forall k \in \mathbb{Z}^+$.
    Therefore
    \begin{align*}
        0 &= \lim_{n \rightarrow \infty} \left(C_{2n+1}-C_{2n}\right) \\
        &= \lim_{n \rightarrow \infty} C_{2n+1}-\lim_{n \rightarrow \infty} C_{2n} \\
        &= \alpha_1 - \alpha_2,
    \end{align*}
    and so $\alpha_1 = \alpha_2$.
\end{proof}

\begin{definition}
    In the theorem above, $\alpha$ is called the value of an \textbf{infinite
    simple continued fraction} $\left[a_0;\,a_1,\,a_2,\,\cdots\right]$.
\end{definition}

\begin{theorem}
    Let $a_0,\,a_1,\,\cdots$ be integers with $a_1,\,a_2,\,\cdots>0$. Then
    $\left[a_0;\,a_1,\,a_2,\,\cdots\right]$ is irrational.
\end{theorem}

\begin{proof}
    Let $\alpha=\left[a_0;\,a_1,\,a_2,\,\cdots\right]$
    and $C_k=\frac{p_k}{q_k}=\left[a_0;\,a_1,\,a_2,\,\cdots,\,a_k\right]$.
    For $n \in \mathbb{Z}^+$,
    \[
        C_{2n} < \alpha < C_{2n+1}
    \]
    and so
    \[
        0 < \alpha - C_{2n} < C_{2n+1} - C_{2n}.
    \]
    Note that 
    \[
        C_{2n+1}-C_{2n}=\frac{1}{q_{2n+1}q_{2n}},
    \]
    thus
    \[
        0<\alpha-C_{2n}=\alpha-\frac{p_{2n}}{q_{2n}} < \frac{1}{q_{2n}q_{2n+1}}.
    \]
    
    Assume that $\alpha$ is rational, so that $\alpha = \frac{a}{b}$ where
    $a,\,b\in\mathbb{Z}$, $b>0$.
    Note that
    \[
        aq_{2n}-bp_{2n} \in \mathbb{Z} \mbox{ for all }n \in \mathbb{N}.
    \]
    Since $q_{2n+1}>2n+1$ for $n \geq 2$, $\exists n_0\in\mathbb{Z}^+$ such that
    \[
        q_{2n_0+1} > b \mbox{ so that } \frac{b}{q_{2n_0}+1} < 1.
    \]
    But this makes a contradiction because
    \[
        0<aq_{2n}-bp_{2n}<\frac{b}{q_{2n+1}}<1.
    \]
\end{proof}
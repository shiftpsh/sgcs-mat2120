\begin{theorem}
    A positive integer $m$ possess a primitive root if and only if
    $m=2,\,4,\,p^t$ or $2p^t$ where $p$ is odd prime and $t \in \mathbb{Z}^+$.
\end{theorem}
\begin{definition}[Index]
    Let $m$ be a positive integer with primitive root $r$. If $a$ is
    an integer with $\left(a,\,m\right)=1$, then $\exists!x$ with $1 \leq x \leq \phi\left(m\right)$
    and $r^x \equiv a \pmod{m}$.
    
    We call $x$ as the \textbf{index} or the \textbf{discrete logarithm}
    of $a$ to the base $r$ modulo $m$. We denote this by
    $\ind_r a$.
\end{definition}
\begin{remark}
    \begin{enumerate}
        \item $r^{\ind_r a} \equiv a \pmod{m}$.
        \footnote{Please don't do such things like $\cancel{a}^{\log_{\cancel{a}} b}=b$, or $\cancel{\sqrt{2^{\cancel{2}}}}=2$. It \underline{ruins} mathematics!}
        \item $\left(a,\,m\right)=1$, $\left(b,\,m\right)=1$, $a \equiv b \pmod{m} \Rightarrow \ind_r a = \ind_r b$.
    \end{enumerate}
\end{remark}

\begin{theorem}
    Let $m$ be a positive integer with primitive root $r$, and $a$ be an integer
    with $\left(a,\,m\right)=1$. Then,
    \begin{enumerate}
        \item $\ind_r 1 \equiv 0 \pmod{\phi\left(m\right)}$.
        \item $\ind_r ab \equiv \ind_r a + \ind_r b \pmod{\phi\left(m\right)}$.
        \item $\ind_r a^k \equiv k \ind_r a \pmod{\phi\left(m\right)}$, where $k \in \mathbb{Z}^+$.
    \end{enumerate}
\end{theorem}

\begin{proof}
    \begin{enumerate}
        \item By Euler's theorem, $r^{\phi\left(m\right)}\equiv 1 \pmod{m}$.
        Since $r$ is a primitive root modulo $m$, no small positive power of $r$ is
        congruent to 1 modulo $m$.
        Hence $\ind_r 1 = \phi\left(m\right) \equiv 0 \pmod{\phi\left(m\right)}$.
        \item Note that
        \[
            r^{\ind_r ab} \equiv ab \pmod{m},
        \] and \[
            r^{\ind_r a + \ind_r b} = r^{\ind_r a}r^{\ind_r b} \equiv ab \pmod{m}.
        \]
        Thus $r^{\ind_r ab} \equiv r^{\ind_r a + \ind_r b}$,
        hence $\ind_r ab \equiv \ind_r a + \ind_r b \pmod{\phi\left(m\right)}$.
        \item Note that
        \begin{align*}
            r^{\ind_r a^k} &\equiv a^k \pmod{m} \\
            r^{k \ind_r a} = \left(a^{\ind_r a}\right)^k &\equiv a^k \pmod{m}
        \end{align*}
        thus $r^{\ind_r a^k} \equiv r^{k \ind_r a} \pmod{m}$, hence
        $\ind_r a^k \equiv k \ind_r a \pmod{\phi\left(m\right)}$.
    \end{enumerate}
\end{proof}

\begin{example}
    $6x^{12} \equiv 11 \pmod{17}$?

    Note that 3 is a primitive root of 17.
    \begin{center}
        \begin{tabular}{c|cccccccccccccccc}
            \hline
            $a$ & 1 & 2 & 3 & 4 & 5 & 6 & 7 & 8 & 9 & 10 & 11 & 12 & 13 & 14 & 15 & 16 \\
            \hline
            $\ind_r a$ & 16 & 14 & 1 & 12 & 5 & 15 & 11 & 10 & 2 & 3 & 7 & 13 & 4 & 9 & 6 & 8 \\
            \hline
        \end{tabular}
    \end{center}
    Taking $\ind_3$ on $6x^2 \equiv 11 \pmod{17}$, we have
    \footnote{
        \begin{align*}
            \because 6x^2 & \equiv 11 \pmod{17} \\
            \Leftrightarrow 3^{\ind_3 6x^2} & \equiv 3^{\ind_3 11} \pmod{17} \\
            \Leftrightarrow \ind_3 6x^2 & \equiv \ind_3 11 \pmod{\phi\left(17\right)}
        \end{align*}
    }
    \[
        \ind_3 \left(6x^{12}\right) \equiv \ind_3 11 \pmod{16}.
    \]
    Thus
    \begin{align*}
        \ind_3 6 + \ind_3 x^{12} & \equiv \ind_3 11 \pmod{16} \\
        \Leftrightarrow 15 + 12 \ind_3 x & \equiv 7 \pmod{16} \\
        \Leftrightarrow 12 \ind_3 x \equiv -8 & \equiv 24 \pmod{16} \\
        \Leftrightarrow \ind_3 x & \equiv 2 \pmod{4}.
    \end{align*}
    Hence $\ind_3 x \equiv 2,\,6,\,10,\,14 \pmod{16}$, therefore,
    $x \equiv 9,\,15,\,8,\,2 \pmod{17}$.

    Since each step in the computation is reversible, there are four
    incongruent solutions to the original equation.

    c.f. If we take another primitive root then we still have the same solution.
\end{example}

\begin{theorem}
    $\left(a,\,m\right)=1$. Suppose that $m$ has a primitive root of $r$.
    Then the following statements are equivalent.
    \begin{enumerate}
        \item $x^n \equiv a \pmod{m}$ has a solution.
        \item $\left(n,\,\phi\left(m\right)\right) \divides \ind_r a$.
        \item $a^{\frac{\phi\left(m\right)}{\left(n,\,\phi\left(m\right)\right)}} \equiv 1 \pmod{m}$
        has $\left(n,\,\phi\left(m\right)\right)$ solutions.
    \end{enumerate}
\end{theorem}

\begin{proof}
    \begin{enumerate}
        \item[$1 \Rightarrow 2$.] Note that
        \begin{align*}
            x^n &\equiv a \pmod{m}
            \Leftrightarrow n \ind_r x &\equiv \ind_r a \pmod{\phi\left(m\right)}.
        \end{align*} 
        Recall that $ax \equiv b \pmod{m}$ has a solution if and only if $\left(a,\,m\right)\divides b$.
        If it has a solution, then it has $\left(a,\,m\right)$ solutions.
        \item[$2 \Leftrightarrow 3$.] Let $c = \phi\left(m\right)$ and $d=\left(n,\,\phi\left(m\right)\right)=\left(n,\,c\right)$.
        Then, $c = dc_1$ for some $c_1$.
        
        Note that\footnote{
            e. g. $x^3 \equiv 4 \pmod{13}$ has no solution.
        }
        \begin{align*}
            a^{\frac{c}{d}} = a^{c_1} &\equiv 1 \pmod{m} \\
            \Leftrightarrow c_1 \ind_r a &\equiv 0 \pmod{\phi\left(m\right)} \\
            \Leftrightarrow c_1d \divides c_1 \ind_r a \\
            \Leftrightarrow d \divides \ind_r a.
        \end{align*}
    \end{enumerate}
\end{proof}

\begin{note}
    Suppose that $m$ has two primitive roots $r$ and $s$. Let $\left(a,\,m\right)=1$.
    Then
     \[
        \ind_s a \equiv \ind_s r \cdot \ind_r a \pmod{\phi\left(m\right)}.
    \]
\end{note}

\begin{proof}
    Put $i=\ind_s a$, $j=\ind_s r$, $k=\ind_r a$. Then
    \[
        s^i \equiv a \quad s^j \equiv r \quad r^k \equiv a \pmod{m},
    \]
    thus
    \[
        s^i \equiv a \equiv r^k \equiv \left(s^j\right)^k \equiv s^{jk} \pmod{m},
    \]
    and so
    \[
        i \equiv jk \pmod{\phi\left(m\right)}.
    \]\qed
\end{proof}

\begin{theorem}[Euler Criterion]
    Let $p$ be an odd prime, and $\left(a,\,p\right)=1$. Then
    $x^2 \equiv a \pmod{p}$
    \begin{enumerate}
        \item has a solution if and only if $p^{\frac{p-1}{2}}\equiv 1 \pmod{p}$.
        \item has no solutions if and only if $p^{\frac{p-1}{2}}\equiv -1 \pmod{p}$.
    \end{enumerate}
\end{theorem}

\begin{proof}
    By Euler's theorem, $a^{p-1} \equiv 1 \pmod{p}$. Then
    \begin{align*}
        a^{p-1}-1 &\equiv 0 \pmod{p} \\
        \Leftrightarrow \left(a^{\frac{p-1}{2}}-1\right)\left(a^{\frac{p-1}{2}}+1\right) &\equiv 0 \pmod{p}.
    \end{align*}
    Note that $1 \not\equiv -1 \pmod{p}$ since $p$ is odd prime.
    By the previous theorem, $x^2 \equiv a \pmod{p}$ has a solution if and only if
    $a^{\frac{p-1}{\left(2,\,p-1\right)}}=a^{\frac{p-1}{2}}\equiv 1 \pmod{p}$.
\end{proof}
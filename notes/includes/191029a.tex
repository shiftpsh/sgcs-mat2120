Consider $ax+b\equiv 0 \pmod{p}$ where $p$ is prime and $p \ndivides a$.
\footnote{If $p\divides a$, then $b \equiv 0 \pmod{p}$.}
Note that this is always solvable.
\footnote{Since $\left(a,\,p\right)=1$, $ax+b\equiv 0 \Rightarrow x \equiv a^{p-2}\left(-b\right) \pmod {p}$.}

Now consider $ax^2+bx+c \equiv 0 \pmod{p}$. Recall $ax^2+bx+c=0$ --
put $x = y + \lambda$. Then
\begin{align*}
    a\left(y+\lambda\right)^2 + b\left(y+\lambda\right) + c &= 0 \\
    \Rightarrow ay^2 + 2a\lambda y + a\lambda^2 + by + b\lambda + c &= 0 \\
    \Rightarrow ay^2 + \left(2a\lambda + b\right)y + a\lambda^2 + b\lambda + c &= 0.
\end{align*}
Put $\lambda = -\frac{b}{2a}$. Then
\begin{align*}
    ay^2 &= -a\lambda^2 - b\lambda - c \\
    &= -a\left(-\frac{b}{2a}\right)^2 - b\left(-\frac{b}{2a}\right) - c \\
    &= \frac{b^2-4ac}{4a}.
\end{align*}
Thus
\begin{align*}
    y^2 &= \frac{b^2-4ac}{4a^2} \\
    \Rightarrow y &= \pm \frac{\sqrt{b^2-4ac}}{2a},
\end{align*}
hence
\[
    x = \frac{-b\pm\sqrt{b^2-4ac}}{2a}.
\]

Return to $ax^2+bx+c \equiv 0 \pmod{p}$. We may assume that $p\ndivides a$ and $p \neq 2$.
\footnote{
    If $p \divides a$, then $ax^2+bx+c \equiv 0 \Leftrightarrow bx+c\equiv 0\pmod{p}$.

    If $p=2$, then $ax^2+bx+c \equiv 0 \pmod{2}$ is too easy.
}
Since $p \ndivides a$, we can find an inverse of $a$, say $a^*$, i. e.
$a \cdot a^* \equiv 1 \pmod{p}$. Then
\[
    ax^2 + bx + c \equiv a\left(x^2 + a^* bx + a^* c\right) \pmod{p}.
\]
Since $p \neq 2$, we can find an inverse $2^* =\frac{p+1}{2}$ for 2 modulo $p$.
Thus
\begin{align*}
    & x^2 + a^* bx + a^* c \\
    & \equiv \left(x+2^* a^* b\right)^2 + a^* c - \left(2^*\right)^2 \left(a^*\right)^2 b^2 \pmod{p}
\end{align*}

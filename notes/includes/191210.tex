
\begin{definition}
    \begin{enumerate}
        \item Stable reduction: no double roots
        \item Multiplicative reduction: double roots
        \item Additive reduction: triple roots.
    \end{enumerate}
    An elliptic curve $E$ is \textbf{semistable} if $E$ has a
    stable reduction or multiplicative reduction.
\end{definition}

Now consider
\[
    \Phi\left(z\right)=
    e^{2\pi iz}\prod_{k=1}^\infty\left(1-e^{8k\pi iz}\right)^2\left(1-e^{16k\pi iz}\right)^2.
\]

Let $q := e^{2\pi iz}$, then
\begin{align*}
    \Phi\left(z\right) &= q \prod_{k=1}^\infty \left(1-q^{4k}\right)^2\left(1-q^{8k}\right)^2\\
    &= q-\underline{2q^5}-3q^9+\underline{6q^13}+\underline{2q^17}+q^25+\cdots.
\end{align*}
It is well known that
\[
    \Phi\left(\frac{az+b}{cz+d}\right)=\left(cz+d\right)^2\Phi\left(z\right)
\]
where $ad-bc=1$; i. e. it is a modular form. Moreover, if we let the
coefficients of $\Phi\left(z\right)$ as $a\left(n\right)$, we get
\begin{center}
    \begin{tabular}{c|cccccccccc}
        \hline
        $F_p$ & $F_2$ & $F_3$ & $F_5$ & $F_7$ & $F_{11}$ & $F_{13}$ & $F_{17}$ & $F_{19}$ & $F_{23}$ & $\cdots$ \\
        \hline
        $S\left(p\right)$ & 2 & 3 & 7 & 7 & 11 & 7 & 15 & 19 & 23 & $\cdots$ \\
        $a\left(p\right)$ & 0 & 0 & $-2$ & 0 & 0 & 6 & 2 & 0 & 0 & $\cdots$ \\
        \hline
    \end{tabular}
\end{center}
where we can see that $S\left(p\right)+a\left(p\right)=p$; hence $y^2=x^3-x$ is modular.

We return to the FLT. Suppose that FLT does not hold, i. e. $\exists$ prime
$p \geq 3$ and $a,\,b,\,c\in\mathbb{Z}^+$ such that $a^p+b^p=c^p$.
Then $\exists$ an elliptic curve $y=x\left(x-a^p\right)\left(x+b^p\right)$.
We call this the Frey curve. (1985 Frey)

It is also well known that the Frey curve is semistable but not modular. (1986 Ribet)
Thus there $\exists$ elliptic curve which is semistable but not modular.

\begin{conjecture}[Taniyama--Shimura]
    Every elliptic curve is modular.
\end{conjecture}

In 1994, Andrew Wiles showed that every semistable elliptic curve is modular, but
this is a contradiction. Hence the FLT holds.

In 2001, Brenil, Conrad, Piamond and Taylor showed that the Taniyana--Shimura
conjecture is a theorem.

\begin{note}
    Consider $\mathbb{Z}\left[\sqrt{-5}\right]
    := \left\{a+b\sqrt{-5}\mid a,\,b\in\mathbb{Z}\right\}$. Then $\mathbb{Z}\left[\sqrt{-5}\right]$ is
    closed under $+$, $-$, $\times$, but is not an unique factorization domain(UFD).
    Examples include $9=3\times 3=\left(2+\sqrt{-5}\right)\left(2-\sqrt{-5}\right)$.
\end{note}

\begin{definition}
    \begin{enumerate}
        \item For any $a,\,b\in\mathbb{Z}\left[\sqrt{-5}\right]$,
        \[
            a\divides b \leftrightarrow \exists c \in \mathbb{Z}\left[\sqrt{-5}\right]
            \quad \mbox{such that} \quad b=ac,
        \]
        i. e. $\mathbb{Z}\left[\sqrt{-5}\right]$ is a \textbf{commutative ring}.
        \item $a\in \mathbb{Z}\left[\sqrt{-5}\right]$. $a$ is said to be an \textbf{unit}
        if $a$ has a multiplicative inverse, i. e. $\exists b \in \mathbb{Z}\left[\sqrt{-5}\right]$
        such that $ab=ba=1$.
        \item $p\in \mathbb{Z}\left[\sqrt{-5}\right]$. $p$ is said to be a \textbf{prime}
        if $p\neq 0$, $p$ is not an unit, and $p \divides ab \Rightarrow p \divides a,\, p \divides b$.
        \item $p$ is said to be an irreducible element if $p \neq 0$, $p$ is not an unit, and
        $p=ab$ where $a$ or $b$ is an unit.
    \end{enumerate}
\end{definition}

\begin{remark}
    In $\mathbb{Z}$, all prime elements are irreducible, and all irreducible elements are prime.
\end{remark}

\begin{note}
    \begin{enumerate}
        \item The set of all units in $\mathbb{Z}\left[\sqrt{-5}\right]=\left\{-1,\,1\right\}$.
        \footnote{
            Define the norm map $f:\mathbb{Z}\left[\sqrt{-5}\right]\rightarrow \mathbb{Z}$
            where $f:a+b\sqrt{-5}\mapsto a^2+5b^2=\left(a+b\sqrt{-5}\right)\left(a-b\sqrt{-5}\right)$.
            It is easy to see that for any $\alpha,\,\beta \in \mathbb{Z}\left[\sqrt{-5}\right]$,
            $f\left(\alpha\right)f\left(\beta\right)=f\left(\alpha\beta\right)$, and $\alpha=0$
            if and only if $f\left(\alpha\right)=0$. 

            Note that $a^2+5b^2=1$ if and only if $a=\pm 1$ and $b=0$. Thus if
            $\left(a+b\sqrt{-5}\right)\left(c+d\sqrt{-5}\right)=1$, then
            $f\left(a+b\sqrt{-5}\right)f\left(c+d\sqrt{-5}\right)=f\left(1\right)$, hence
            $a^2+5b^2=c^2+5d^2=1$ and so $a=c=\pm 1$ and $b=d=0$.
        }
        \item 3, $2-\sqrt{-5}$, $2+\sqrt{-5}$ are irreducible elements.
        \footnote{
            Clearly 3, $2-\sqrt{-5}$, $2+\sqrt{-5}$ are nonzero and not an unit. Note that
            \begin{align*}
                f\left(3\right) = 9 &= f\left(2+\sqrt{-5}\right) \\
                &= f\left(2-\sqrt{-5}\right).
            \end{align*}
            If $z=a+b\sqrt{-5}$ is a factor among 3, $2-\sqrt{-5}$, $2+\sqrt{-5}$,
            then $f\left(z\right)=1$, 3, or 9. Suppose $f\left(z\right)=3$, then
            $a^2+5b^2=3$, which is a contradiction.
        }
    \end{enumerate}
\end{note}
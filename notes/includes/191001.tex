
\begin{definition}
    $m \in \mathbb{Z}^+$, $a,\,b \in \mathbb{Z}$

    $a$ is \textbf{congrugent} to $b$ modulo $m$ if $m \divides \left(a-b\right)$.
\end{definition}

\begin{theorem}
    \begin{enumerate}
        \item $a \equiv a$ (mod $m$)
        \item $a \equiv b \Rightarrow b \equiv a$
        \item $a \equiv b,\, b \equiv c \Rightarrow a \equiv c$
        \item $a \equiv b,\, c \equiv d \Rightarrow a \pm c \equiv b \pm d,\, ac \equiv bd$.
        \item[$\mathit{4\frac{1}{2}}$.] $1 \leq i \leq n$. Then
        $a_i \equiv b_i \Rightarrow \sum_1^n a_i \equiv \sum_1^n b_i,\, \prod_1^n a_i \equiv \prod_1^n b_i$ 
        \item Let $f\left(x\right) = a_0 + a_1x + \cdots + a_nx^n$, $g\left(x\right) = b_0 + b_1x + \cdots + b_nx^n$,
        where $a_i,\,b_i \in \mathbb{Z}$. Suppose $a_i \equiv b_i$ (mod $m$). If $a \equiv b$, then $f\left(a\right) \equiv g\left(b\right)$.
    \end{enumerate}
\end{theorem}

\begin{example}
    $10 \equiv 1$ (mod 3).

    $10 \equiv 1$ (mod 9).
    
    $10 \equiv -1$ (mod 11).

    Let $a=a_n \cdot 10^n+\cdots+a_1 \cdot 10+a_0$. Then
    \begin{align*}
        a &\equiv a_0 + a_1 + \cdots + a_n \mbox{ (mod 3)} \\
        &\equiv a_0 + a_1 + \cdots + a_n \mbox{ (mod 9)} \\
        &\equiv a_0 - a_1 + \cdots + \left(-1\right)^n a_n \mbox{ (mod 11)} \\
    \end{align*}

    $\because$ If $f\left(x\right)=a_0+a_1x+\cdots+a_nx^n$, then
    \begin{align*}
        f\left(10\right) &\equiv f\left(1\right) \mbox{ (mod 3)} \\
        f\left(10\right) &\equiv f\left(1\right) \mbox{ (mod 9)} \\
        f\left(10\right) &\equiv f\left(-1\right) \mbox{ (mod 11)} \\
    \end{align*}

    e. g. 
    \begin{align*}
        26384 &\equiv 2+6+3+8+4 \equiv 2 \mbox{ (mod 3)} \\
        26384 &\equiv 2+6+3+8+4 \equiv 5 \mbox{ (mod 9)} \\
        26384 &\equiv 2-6+3-8+4 \equiv 6 \mbox{ (mod 11)} \\
    \end{align*}
\end{example}

\begin{example}
    $41 \divides \left(2^{20}-1\right)$?

    Note that
    \[
        2^5 \equiv -9 \mbox{ (mod 41)}.  
    \]
    Thus
    \begin{align*}
        \left(2^5\right)^4 &\equiv \left(-9\right)^4 \\
        &\equiv 81 \times 81
    \end{align*}
    Since $81 \equiv -1 \mbox{ (mod 41)}$, $81\times 81 \equiv 1 \mbox{ (mod 41)}$.
    Hence
    \begin{align*}
        2^{20}-1 &\equiv\left(2^5-4\right)-1 \\
        &\equiv \left(-9 \right)^4-1 \\
        &\equiv 1 - 1 \equiv 0 \mbox{ (mod 41)}.
    \end{align*}
\end{example}

Note that $7\times 2 \equiv 4 \times 2$ (mod 6), but $7 \not\equiv 4$ (mod 6),
also $7 \equiv 4$ (mod 3).

\begin{theorem}
    $a,\,b,\,c \in \mathbb{Z}$, $m \in \mathbb{Z}^+$,\, $d=\left(c,\,m\right)$.

    If $ac \equiv bc$ (mod $m$), then $a \equiv b$ (mod $\frac{m}{d}$).
\end{theorem}

\begin{proof}
    Since $ac \equiv bc$ (mod $m$),
    \[
        m \divides \left(ac-bc\right).    
    \]

    Thus $\exists k \in \mathbb{Z}$ such that $c\left(a-b\right)=km$,
    and so
    \[
        \frac{c}{d} \left(a-b\right) = k \frac{m}{d}.        
    \]

    Since $\left(\frac{c}{d},\,\frac{m}{d}\right)=1$, it follows that
    \[
        \frac{m}{d} \divides \left(a-b\right).    
    \]
    \qed
\end{proof}

\paragraph{Question.} $2^{1137} \equiv \mbox{?}$ (mod 17)

\begin{theorem}
    Let $m\in \mathbb{Z}^+$. For any $a\in \mathbb{Z}$, $\exists! r \in \mathbb{Z}$
    such that
    \[
        a \equiv r \mbox{ (mod $m$)}
    \]
    where $0 \leq r \leq m - 1$.
\end{theorem}

\begin{proof}
    Use the division algorithm.
\end{proof}

\begin{definition}
    A \textbf{complete system of residues} modulo $m$
    is the set of integers such that every integers is congrugent modulo $m$
    to exactly one integer of the set.
\end{definition}

e. g.
\begin{enumerate}
    \item $\left\{0,\,1,\,2,\,\cdots,\,m-1\right\}$
    is a complete system of residues modulo $m$.\footnote{The least nonnegative residues modulo $m$}
    \item If $m$ is odd,
    $\left\{-\frac{m-1}{2},\,-\frac{m-3}{2},\,\cdots,\,-1,\,0,\,1,\,\cdots,\,\frac{m-3}{2},\,\frac{m-1}{2}\right\}$
    is also a complete system of residues modulo $m$.
\end{enumerate}

\begin{theorem}
    If $\left\{r_1,\,r_2,\,\cdots,\,r_m\right\}$ is a complete system of residues
    modulo $m$ and if $a\in \mathbb{Z}^+$ with \underline{$\left(a,\,m\right) = 1$},
    then for any integer $b$,
    \[
        \left\{ar_1+b,\,ar_2+b,\,\cdots,\,ar_m+b\right\}
    \]
    is a complete system of residues modulo $m$.
\end{theorem}
e. g. $m=4 \Rightarrow \left\{0,\,1,\,2,\,3\right\},\, \left\{0,\,3,\,6,\,9\right\},\,
\left\{1,\,2,\,3,\,4\right\},\, \cdots$

but $\left\{0,\,2,\,4,\,6\right\}$ is not a complete system of residues modulo 4.

\begin{proof}
    Note that a set of $m$ incongrugent integers modulo $m$
    will always form a complete system of residues modulo $m$.

    Thus it suffices to show that no two integers $ar_1+b,\,\cdots,\,ar_m+b$
    are congrugent modulo $m$.

    Suppose that
    \[
        ar_j+b \equiv ar_k+b.    
    \]
    then
    \[
        ar_j \equiv ar_k.    
    \]

    Since $\left(a,\,m\right)=1$, $r_j\equiv r_k$. Hence $j=k$. \qed
\end{proof}

\begin{theorem}
    $a,\,b \in \mathbb{Z}^+$, $m \in \mathbb{Z}^+$, $d=\left(a,\,m\right)$.

    If $d \ndivides b$, then $ax \equiv b$ (mod $m$) has no solutions.

    If $d \divides b$, then $ax \equiv b$ (mod $m$) has exactly $d$ incongrugent
    solutions modulo $m$ as follows:
    \[
        x = x_0 + \frac{m}{d}t \qquad t=0,\,1,\,2,\,\cdots,\,d-1    
    \]
    where $x_0$ is a particular solution of $ax\equiv b$ (mod $m$).
\end{theorem}

\begin{example}
    $9x \equiv 12$ (mod 15)?

    Note that $\left(9,\,15\right)=3\divides 12$, by theorem,
    $\exists$ exactly 3 incongrugent solutions modulo 15.

    To find a particular solution, consider $9x+15y=12$.
    Note that
    \begin{align*}
        15&=9 \times 1 + 6 \\
        9&=6 \times 1 + 3 \\
        6&=3 \times 2 + 0 \\
        3&=9-6=9\times 2 - 15.
    \end{align*}
    Thus $9 \times 8 + 15 \times \left(-4\right) = 12$.

    Hence the general solution is given by
    \begin{align*}
        x&=x_0\equiv8 \mbox{ (mod 15)} \\ 
        x&=x_0+\frac{15}{3}\times 1\equiv13 \mbox{ (mod 15)} \\ 
        x&=x_0+\frac{15}{3}\times 2=18\equiv 3 \mbox{ (mod 15)}.
    \end{align*}
\end{example}

\begin{proof}
    (Proof left for homework -- due October 3rd.)
\end{proof}

\begin{remark}
    Consider $ax \equiv 1$ (mod $m$).
    By the previous theorem, $\exists$ solutions of this congrugence
    if and only if $\left(a,\,m\right) = 1$.
\end{remark}

\begin{definition}
    $a \in \mathbb{Z}$, $m \in \mathbb{Z}^+$, $\left(a,\,m\right)=1$.

    A solution of $ax \equiv 1$ (mod $m$) is called an \textbf{inverse} of
    $a$ modulo $m$.
\end{definition}

e. g. $7x\equiv 1$ (mod 31) $\Rightarrow$ $x=9$ (mod 31).
Thus 9 and all integers congrugent to 9 are inverses of 7 modulo 31.

e. g. $7x \equiv 22$ (mod 31) $\Rightarrow$ $9\times 7x \equiv 9\times 22$ (mod 31)
$\Rightarrow$ $1\times x \equiv 12$ (mod 31)

\begin{remark}
    $\mathbb{Z}_n^*=\left\{\overline{a}\in \mathbb{Z}_m \mid \left(a,\,m\right)=1\right\}$.
\end{remark}
\marginpar{
    $\mathbb{Z}_5=\left\{\overline{0},\,\overline{1},\,\overline{2},\,\overline{3},\,\overline{4}\right\}$
}

e. g. $\mathbb{Z}_{8}^*=\left\{\overline{1},\,\overline{3},\,\overline{5},\,\overline{7}\right\}$
\begin{lemma}
    $\alpha$ : quadratic irrational. $r,\,s,\,t,\,u \in \mathbb{Z}$, then
    $\frac{r\alpha+s}{t\alpha+u}$ is rational or quadratic irrational.    
\end{lemma}
\begin{proof}
    By previous lemma,
    \[
        \alpha=\frac{a+\sqrt{b}}{c} \mbox{ where } \cdots
    \]
    then
    \begin{align*}
        \frac{r\alpha+s}{t\alpha+u} &= \frac{r\left(\frac{a+\sqrt{b}}{c}\right)+s}{t\left(\frac{a+\sqrt{b}}{c}\right)+u} \\
        &= \frac{ar + cs + r\sqrt{b}}{at + cu + t\sqrt{b}} \\
        &= \frac{\left(\qquad\right)+\left(\qquad\right)\sqrt{b}}{\left(at+cu\right)^2-t^2b} \mbox{ (see the book for detailed calculation.)}
    \end{align*}
\end{proof}

\begin{theorem}
    An infinite simple continued fraction of $\alpha$ is periodic if and only if
    $\alpha$ is quadratic irrational.
\end{theorem}

e. g. $x=\left[ 3;\,\overline{1,\,2} \right]$

$\Rightarrow y=\left[ \overline{1;\,2} \right] \Rightarrow y = \left[ 1;\,2,\,y \right] = 1 + \frac{1}{2 + \frac{1}{y}}$

\begin{proof}
    ($\Rightarrow$, by Euler in 1737) Let
    \[\alpha=\left[ a_0;\,a_1,\,a_2,\,\cdots,\,a_{N-1},\,a_{N},\,a_{N+1},\,\cdots,\,a_{N+k} \right]\]
    and
    \[\beta=\left[ a_{N};\,a_{N+1},\,\cdots,\,a_{N+k} \right].\]
    Then
    \[\beta=\left[ a_{N};\,a_{N+1},\,\cdots,\,a_{N+k},\,\beta \right].\]
    By Theorem 12.9.,
    \begin{align}
        \beta=\frac{\beta p_k+p_{k-1}}{\beta q_k+q_{k-1}}
    \end{align}
    where $\frac{p_k}{q_k}$, $\frac{p_{k+1}}{q_{k+1}}$ are convergents of $\beta$.
    Clearly\footnote{$\beta$ is represented as an infinite simple continued fraction.},
    $\beta$ is irrational.

    By (1),
    \[
        q_k \beta^2 + \left(q_{k-1}-p_k\right)\beta - p_k=0,
    \]
    thus $\beta$ is a quadratic irrational.

    Note that
    \[
        \alpha = \left[ a_0;\,a_1,\,a_2,\,\cdots,\,a_{N-1},\,\beta \right].
    \]
    By theorem,
    \[
        \alpha = \frac{\beta p_{N-1}+p_{N-2}}{\beta q_{N-1}+q_{N-2}} 
    \]
    where $\frac{p_{N-1}}{q_{N-1}}$, $\frac{p_{k+{N-2}}}{q_{k+{N-2}}}$ are convergents of $\alpha$.
    Hence $\alpha$ is quadratic irrational, and it represents an infinite simple continued fraction.

    ($\Leftarrow$, by Lagrange in 1770)
    \begin{lemma}
        If $\alpha$ is a quadratic irrational, then $\alpha$ can be written as
        \[
            \alpha = \frac{P+\sqrt{d}}{Q}
        \]
        where $P,\,Q \in \mathbb{Z}$, $Q\neq 0$, $d > 0$, $d$ is not perfect square,
        and $Q \divides \left(d-P^2\right)$.
    \end{lemma}
    \begin{proof}[of Lemma]
        By Lemma,
        \[
            \alpha = \frac{a+\sqrt{b}}{c}.
        \]
        Put $P:=a\left|c\right|$, $Q:=c\left|c\right|$, $d=bc^2$. Then
        $P,\,Q,\,d\in\mathbb{Z}$, $d>0$, and $d$ is not perfect square since $b$ is also not.

        Note that
        \begin{align*}
            d-p^2 &= bc^2 - a^2c^2 \\
            &= \left(b-a^2\right)c^2 \\
            &= \pm \left(b-a^2\right)Q,
        \end{align*}
        thus $Q \divides \left(d-p^2\right)$.
    \end{proof}
    \begin{theorem}
        $\alpha$ : quadratic irrational. By Lemma,
        \[
            \alpha_0=\alpha=\frac{P_0+\sqrt{d}}{Q_0}.
        \]
        Define $\alpha_k:=\frac{a_k+\sqrt{d}}{Q_k}$, $a_k:=\left\lfloor \alpha_k\right\rfloor$,
        $P_{k+1}:=a_kQ_k-P_k$, $Q_{k+1}:=\frac{a-P_{k+1}^2}{Q_k}$ for $k=0,\,1,\,2,\,\cdots$.
        Then
        \[
            \alpha=\left[a_0;\,a_1,\,a_2,\,\cdots\right].
        \]
    \end{theorem}
    \begin{proof}
        \underline{Claim} $P_k,\,Q_k \in \mathbb{Z}$ with $Q_k \neq 0$ and $Q_k \divides \left(d-p_k^2\right)$.

        Note that the induction base is true. Now suppose that 
        $P_k,\,Q_k \in \mathbb{Z}$ with $Q_k \neq 0$ and $Q_k \divides \left(d-P_k\right)^2$. Then
        \begin{align*}
            P_{k+1} &= a_kQ_k-P_k \in \mathbb{Z} \\
            Q_{k+1} &= \frac{d-P_{k+1}^2}{Q_k} \\
            &= \frac{d-\left(a_kQ_k - P_k\right)^2}{Q_k} \\
            &= \frac{d-P_k^2}{Q_k} + 2a_kP_k - a_k^2Q_k.
        \end{align*}
        By the induction hypothesis, $Q_{k+1} \in \mathbb{Z}$ and $Q_{k+1} \divides \left(d-P_{k+1}^2\right)$.

        \underline{Claim} $\alpha = \left[ a_0;\,a_1,\,a_2,\,\cdots \right]$

        Suffices to show that
        \[
            \alpha_{k+1}=\frac{1}{\alpha_k-a_k} \mbox{ for } k=0,\,1,\,2,\,\cdots.
        \]
        Note that
        \begin{align*}
            \alpha_k-a_k &= \frac{p_k+\sqrt{d}}{Q_k}-a_k \\
            &= \frac{\sqrt{d}-\left(a_kQ_k-P_k\right)}{Q_k} \\ 
            &= \frac{\sqrt{d}-P_{k+1}}{Q_k} \\
            &= \frac{\left( \sqrt{d}-P_{k+1} \right)\left( \sqrt{d}+P_{k+1} \right)}{Q_k\left( \sqrt{d}-P_{k+1} \right)} \\
            &= \frac{d-P_{k+1}^2)}{Q_k\left( \sqrt{d}-P_{k+1} \right)} \\
            &= \frac{Q_k+1}{\sqrt{d}+P_{k+1}} \\
            &= \frac{1}{\alpha_{k+1}}.
        \end{align*}
    \end{proof}

    We return to our original proof ($\Leftarrow$).

    By the previous theorem,
    $@a = \left[ a_0;\,a_1,\,a_2,\,\cdots \right]$ where $a_k=\left\lfloor\frac{P_k+\sqrt{d}}{Q_k}\right\rfloor$.
    Note that
    \[
        \alpha=\left[ a_0;\,a_1,\,a_2,\,\cdots,\,a_{k-1},\,\alpha_k \right],
    \]
    thus $\alpha=\frac{p_{k-1}\alpha_k+p_{k-2}}{q_{k-1}\alpha_k+q_{k-2}}$ by theorem.

    \begin{definition}
        $\alpha = \frac{a+\sqrt{b}}{c}$; Define $\alpha^\prime:=\frac{a-\sqrt{b}}{c}$.
    \end{definition}

    By taking conjugates,
    \[
        \alpha^\prime = \frac{p_{k-1}\alpha^\prime_k+p_{k-2}}{q_{k-1}\alpha^\prime_k+q_{k-2}}.
    \]
    Then
    \[
        \alpha_k^\prime = -\frac{q_{k-2}}{q_{k-1}}
        \left(\frac
            {\alpha^\prime -\frac{p_{k-2}}{q_{k-2}}}
            {\alpha^\prime -\frac{p_{k-1}}{q_{k-1}}}
        \right).
    \]
    Note that $\frac{p_{k-2}}{q_{k-2}} \rightarrow \alpha$,
    $\frac{p_{k-1}}{q_{k-1}} \rightarrow \alpha$ as $k \rightarrow \infty$. And so,
    $\frac
    {\alpha^\prime -\frac{p_{k-2}}{q_{k-2}}}
    {\alpha^\prime -\frac{p_{k-1}}{q_{k-1}}} \rightarrow 1$.

    Thus $\exists N$ such thst $\alpha_k^\prime < 0$ for $k \geq N$.

    Since $\alpha_k>0$ for $k \geq 1$($\because a_k:=\left\lfloor\alpha_k\right\rfloor>0$), 
    it follows that
    \begin{align*}
        \alpha_k-\alpha_k^\prime &= \frac{P_k+\sqrt{d}}{Q_k} - \frac{P_k-\sqrt{d}}{Q_k} \\
        &= \frac{2\sqrt{k}}{Q_k} > 0 \mbox{ for } k \geq N.
    \end{align*}
    Thus $Q_k>0$ for $k \geq N$.

    Since $Q_kQ_{k+1} = d-P_{k+1}^2$, it follows that for $k \geq N$, $Q_k \leq Q_kQ_{k+1} = d-P_{k+1}^2<d$.

    Also for $k \geq N$,
    \[
        P_{k+1}^2 < d = \left(P_{k+1}^2+Q_kQ_{k+1}\right)
    \]
    and so $-\sqrt{d}<P_{k+1}<\sqrt{d}$.
\end{proof}
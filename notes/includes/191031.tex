Put $y=x+2^* a^* b$ and $d=\left(2^*\right)^2 \left(a^*\right)^2 b^2-a^* c$.
Hence
\[
    ax^2 + bx + c \equiv 0 \pmod{p} \quad \mbox{ if and only if } \quad y^2 \equiv d \pmod{p}.
\]

\begin{remark}
    $ax^2+bx+c \equiv 0 \pmod{p}$, $p$ is odd prime, $p \ndivides a$.
    Since $p$ is odd and $p \ndivides a$, $p \ndivides 4a$. Thus
    \[
        ax^2 + bx + c \equiv 0 \pmod{p} \quad \mbox{ if and only if } \quad 4a\left(ax^2 + bx + c\right) \equiv 0 \pmod{p}.
    \]
    Note that
    \begin{align*}
        & 4a\left(ax^2 + bx + c\right) \\
        &= 4a^2x^2 + 4abx + 4ac \\
        &= \left(2ax+b\right)^2 - \left(b^2-4ac\right).
    \end{align*}
    Thus
    \[
        ax^2+bx+c \equiv 0 \pmod{p} \quad \mbox{ if and only if } \quad y^2 \equiv A \pmod{p}
    \]
    where $y=2ax+b$ and $A=b^2-4ac$.
\end{remark}

\begin{example}
    $3x^2 - 4x + 7 \equiv 0 \pmod{13}$?

    \begin{enumerate}
        \item (calculation left for students.)
        \item $4a = 4 \cdot 3 = 12$. We solve $12\left(3x^2 - 4x + 7\right) \equiv 0 \pmod{13}$.
        Then
        \begin{align*}
            \left(6x-4\right)^2 &\equiv 16-84 \pmod{13} \\
            &\equiv 10 \pmod{13} \\
        \end{align*}
        Put $y=6x-4$. Then $t^2\equiv 10 \pmod{13}$. It can be seen that
        $\exists$ two solutions $y \equiv 6$ or $7 \pmod{13}$.\footnote{
            $2 \ind_6 y \equiv \ind_6 10 \pmod{12} \Rightarrow \ind_6 y \equiv 1$ or $7 \pmod{12}$
        }
        Note that
        \begin{align*}
            6x-4 &\equiv 6 \pmod{13} \\
            6x-4 &\equiv 7 \pmod{13}
        \end{align*}
        Hence $x \equiv 6$ or $x \equiv 4 \pmod{13}$.
    \end{enumerate}
\end{example}

\begin{definition}[Quadratic Residue]
    $m \in \mathbb{Z}^+$, $\left(a,\,m\right)=1$.
    We say that $a$ is a \textbf{quadratic residue} of $m$ if
    \[
        x^2 \equiv a \pmod{m}
    \]
    has a solution.
\end{definition}

\begin{remark}
    $x^2 \equiv a \pmod{p}$ and $p$ is prime. If $\left(p,\,a\right)=1$, i. e. $\left(i,\,p\right)\neq 1$,
    then $x \equiv 0$ is the only solution.\footnote{
        Since $p\divides a$, $a \equiv 0 \pmod{p}$. Then $x^2 \equiv a \equiv 0 \pmod{p}$.

        If $x_0$ is a solution, then $p \divides x_0^2$ and so $p \divides x_0$, thus $x_0 \equiv 0 \pmod{p}$.
    }
\end{remark}

\begin{example}
    $p=13$. Note that
    \begin{align*}
        1^2 \equiv 1 &\equiv 12^2 \pmod{13} \\
        2^2 \equiv 4 &\equiv 11^2 \pmod{13} \\
        3^2 \equiv 9 &\equiv 10^2 \pmod{13} \\
        4^2 \equiv 3 &\equiv 9^2 \pmod{13} \\
        5^2 \equiv 12 &\equiv 8^2 \pmod{13} \\
        6^2 \equiv 10 &\equiv 7^2 \pmod{13}
    \end{align*}
    Thus 13 has exactly 6 quadratic residues, namely 1, 3, 4, 9, 10, 12, and 6 quadratic non-residues,
    2, 5, 6, 7, 8, 11.\footnote{
        FYI: In 1973, R. H. Hudson proved that 13 is the only prime $p$ that has more than $\sqrt{p}$
        consecutive quadratic nonresidues.
    }   
\end{example}

\begin{lemma}
    $p$ : odd prime, $p \ndivides a$. Then $x^2 \equiv a \pmod{p}$ has either
    no solutions, or exactly 2 incongrugent solutions modulo $p$.
\end{lemma}

\begin{proof}
    If $x^2 \equiv a \pmod{p}$ has a solution, say $x=x_0$. Then
    $x \equiv -x_0 \equiv p-x_0$ is also a solution.

    Note that $x_0 \not\equiv -x_0 \pmod{p}$.\footnote{
        If $x_0 \equiv -x_0$, then $2x_0 \equiv 0$. Note that
        $x_0^2 \equiv a \Rightarrow p\ndivides x_0,\, p\ndivides 2 \Rightarrow p\ndivides 2x_0$.
    }
    To show that there are no more than 2 incongrugent solutions, assume that
    $x\equiv x_0$ and $x\equiv x_1$ are both somutions of $x^2 \equiv a \pmod{p}$. Note that
    \[
        x_0^2 \equiv x_1^2 \equiv a \quad \mbox{ and so } \quad 0 \equiv x_0^2-x_1^2 \equiv
        \left(x_0-x_1\right)\left(x_0+x_1\right).
    \]
    This $p\divides \left(x_0-x_1\right)$ or $p \divides \left(x_0+x_1\right)$, hence
    $x_0 \equiv x_1 \pmod{p}$ or $x_0 \equiv -x_1 \pmod{p}$. \qed
\end{proof}

\begin{theorem}
    Every odd prime $p$ has exactly $\frac{p-1}{2}$ quadratic residues and $\frac{p-1}{2}$
    quadratic nonresidues.
\end{theorem}

\begin{proof}
    (Using Lemma; See the book.)
\end{proof}

\begin{definition}[Legendre Symbol]
    $p$ : odd prime, $p \ndivides a$. Then
    \[
        \legendre{a}{p} := \begin{cases}
            1 & \mbox{if $a$ is a quadratic residue modulo $p$} \\
            -1 & \mbox{otherwise}
        \end{cases}
    \]
\end{definition}

e. g.
\begin{align*}
    \legendre{1}{11} = \legendre{3}{11} = \legendre{4}{11} = \legendre{5}{11} = \legendre{9}{11} &= 1 \\
    \legendre{2}{11} = \legendre{6}{11} = \legendre{7}{11} = \legendre{8}{11} = \legendre{10}{11} &= -1
\end{align*}

Question: $\legendre{713}{1009}=?$

$x^2 \equiv 713 \pmod{1009}?$

\paragraph{Recall} Euler's criterion

$p$ : odd prime, $p \ndivides a$. $x^2 \equiv a \pmod{p}$ has a solution if and only if
$a^{\frac{p-1}{2}} \equiv 1 \pmod{p}$.

\begin{theorem}[Euler Criterion]
    $p$ : odd prime, $p\ndivides a$. Then

    \[
        \legendre{a}{p} \equiv a^{\frac{p-1}{2}} \pmod{p}.
    \]
\end{theorem}

e. g.
\begin{align*}
    \legendre{5}{23} &\equiv 5^{\frac{23-1}{2}} \pmod{23} \\
    &\equiv 5^{11} \pmod{23} \\
    &\equiv \left(5^2\right)^5 \cdot 5 \pmod{23} \\
    &\equiv 2^5 \cdot 5\pmod{23} \\
    &\equiv -1 \pmod{23}.
\end{align*}
Hence $\legendre{5}{23}=-1$.

\begin{theorem}
    $p$ : odd prime, $p \ndivides a$, $p \ndivides b$.

    \begin{enumerate}
        \item $a \equiv b \pmod{p} \Rightarrow \legendre{a}{p} = \legendre{b}{p}$.
        \item $\legendre{a}{p}\legendre{b}{p} = \legendre{ab}{p}$.
        \item $\legendre{a^2}{p}=1$.
    \end{enumerate}
\end{theorem}

\begin{proof}
    \begin{enumerate}
        \item[1., 3.] It's trivial.
        \item[2.] {
            By Euler criterion,
            \[
                \legendre{a}{p}\equiv a^{\frac{p-1}{2}},\qquad
                \legendre{b}{p}\equiv b^{\frac{p-1}{2}},
            \]
            and
            \[
                \legendre{ab}{p}\equiv ab^{\frac{p-1}{2}}.
            \]
            Thus
            \[
                \legendre{a}{p}\legendre{b}{p} \equiv a^{\frac{p-1}{2}}b^{\frac{p-1}{2}}
                \equiv ab^{\frac{p-1}{2}} \equiv\legendre{ab}{p}
            \]
            hence $\legendre{a}{p}\legendre{b}{p} = \legendre{ab}{p}$.
        } 
    \end{enumerate}
\end{proof}

\paragraph{Proposition.} $p$ : odd prime.
\[
    \legendre{-1}{p} = \begin{cases}
        1 & \mbox{if $p \equiv 1 \pmod{4}$} \\
        -1 & \mbox{if $p \equiv 3 \pmod{4}$}
    \end{cases}
\]

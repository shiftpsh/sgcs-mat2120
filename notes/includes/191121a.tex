
\paragraph{Note} It is known that the ancient Greeks studied the Diophantine equation $x^2-2y^2=1$ to
try to understand $\sqrt{2}$. Note that
\[
    \frac{x^2}{y^2} = 2+\frac{1}{y^2} \Rightarrow
    \sqrt{2+\frac{1}{y^2}} \rightarrow \frac{x}{y} = \sqrt{2} \mbox{ as } y \rightarrow \infty.
\]

Consider $x^2-dy^2=n$ where $d,\,n \in \mathbb{Z}$.
\begin{itemize}
    \item When $d<0$ and $n<0$, there are no solutions.
    \item When $d<0$ and $n<0$, there can be at most a finitely many solutions.
    \footnote{$\because x^2-dy^2=n \Rightarrow \left|x\right|\leq \sqrt{n}$;
    $\left|y\right|\leq\sqrt\frac{n}{\left|d\right|}$}
    \item When $d$ is a perfect square, say $d=D^2$, then $x^2-dy^2=x^2-D^2y^2
    =\left(x-Dy\right)\left(x+Dy\right)$. Thus any solution of $x^2-dy^2=n$ corresponds to
    a simultaneous solution of $\begin{cases}
        x+Dy=a \\
        x-Dy=b
    \end{cases}$ where $n=ab$ and $a,\,b\in\mathbb{Z}$. Hence there are only finitely many solutions.
\end{itemize}

Now we are interested in the case when $x^2-dy^2=n$ where $d,\,n\in\mathbb{Z}$, and $d>0$ is
not a perfect square.

\begin{theorem}
    If $x$ is an irrational number and $\frac{r}{s}$ is a rational number in lowest terms where
    $r,\,s\in\mathbb{Z}$, $s>0$, such that $\left|\alpha-\frac{r}{s}\right|<\frac{1}{2s^2}$ then
    $\frac{r}{s}$ is a convergent of the simple continued fraction of $\alpha$.
\end{theorem}
\begin{proof}
    (See the book.)
\end{proof}

\begin{theorem}
    Let $d,\,n\in\mathbb{Z}$, $d>0$ is anot a perfect square, and $\left|n\right|\leq\sqrt{d}$.
    If $x^2-dy^2=n$, $x>0$, $y>0$, then $\frac{x}{y}$ is a convergent of the simple continued
    fraction of $\sqrt{d}$.
\end{theorem}

\begin{proof}
    We only consider the case where $n>0$. Note that
    \[x^2-dy^2=n \Leftrightarrow \left(x+y\sqrt{d}\right)\left(x-y\sqrt{d}\right)=n,\]
    thus $x-y\sqrt{d}>0$, i. e. $x>y\sqrt{d} \Rightarrow \frac{x}{y}>\sqrt{d}$.

    Note that
    \begin{align*}
        \frac{x}{y}-\sqrt{d}&=\frac{x-y\sqrt{d}}{y}\frac{x+y\sqrt{d}}{x+y\sqrt{d}}\\
        &= \frac{x^2-y^2d}{y\left(x+y\sqrt{d}\right)} \\
        &< \frac{n}{y\left(2y\sqrt{d}\right)} \\
        &< \frac{\sqrt{d}}{2y^2\sqrt{d}} = \frac{1}{2y^2}.
    \end{align*}

    By previous theorem, we see that $\frac{x}{y}$ is a convergent of the
    simple continued fraction of $\sqrt{d}$.
\end{proof}

\begin{theorem}
    $d>0$, $d\in\mathbb{Z}$ and is not a perfect square.

    $\frac{p_k}{q_k}$ : the $k$-th convergent of the simple continued fraction of $\sqrt{d}$.

    $n$ : the periodic length of the simple continued fraction of $\sqrt{d}$. Then,
    \begin{enumerate}
        \item If $n$ is even, the positive solution of $x^2-dy^2=1$ are $x=p_{jn-1}$, $y=q_{jn-1}$
        where $j\in\mathbb{N}$; and $x^2-dy^2=-1$ has no solutions.
        \item If $n$ is odd, the positive solutions of $x2-dy^2=1$ are $x=p_{2jn-1}$, $y=q_{2jn-1}$
        where $j\in\mathbb{N}$, and that of $x^2-dy^2=-1$ are $x=p_{\left(2j-1\right)n-1}$,
        $y=q_{\left(2j-1\right)n-1}$.
    \end{enumerate}
\end{theorem}
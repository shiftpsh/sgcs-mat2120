Thus we can see that $\exists$ only a finite number of pussible values
for $P_k$ and $Q_k$ for $k \geq N$.

Since $\exists$ infinitely many integers $k$ with $k \geq N$, $\exists$
two integers $i$ and $j$ such that $P_i=P_j$ and $Q_i=Q_j$.
By the defining relation for $\alpha_k$, we can see that $\alpha_i=\alpha_j$.
Consequently, we can see that $a_i=a_j$ and $a_{i+1}=a_{j+1}$.
Hence
\begin{align*}
    \alpha&=\left[ a_0;\,a_1,\,a_2,\,\cdots,\,a_{i-1},\,a_i,\,a_{i+1},\,\cdots,\,a_{j-1},\,a_i,\,\cdots \right]\\
    &= \left[ a_0;\,a_!,\,a_2,\,\cdots,\,a_{i-1},\,\overline{a_i,\,\cdots,\,a_{j-1}} \right].
\end{align*}

\begin{definition}
    A simple continued fraction $\left[ a_0;\,a_1,\,a_2,\,\cdots \right]$ is
    called \textbf{purely periodic} if $\exists n \in \mathbb{Z}^+$ such that
    $a_k=a_{k+n}$ for $k=0,\,1,\,2,\,\cdots$. i. e.,
    \[
        \left[ a_0;\,a_1,\,a_2,\,\cdots \right] = \left[ \overline{a_0;\,a_1,\,a_2,\,\cdots,\,a_{n-1}} \right]
    \]
\end{definition}

e. g. $\left[ \overline{2;\,3} \right]=\frac{3+\sqrt{15}}{3}$ is purely periodic.

$\left[ 2;\,\overline{2,\,4} \right]=\sqrt{6}$ is not purely periodic.

\begin{definition}
    A quadratic irrational $\alpha$ is said to be \textbf{reduced} if
    $\alpha > 1$ and $-1 < \alpha^\prime < 0$.
\end{definition}

\begin{theorem}
    The simple continued fraction of quadratic irrational number $\alpha$
    is purely periodic if and only if $\alpha$ is reduced.
\end{theorem}

\begin{proof}
    (Proof omitted.)
\end{proof}

\begin{remark}
    Consider $\sqrt{D}$, where $D$ is not a perfect square. Note that
    $\left\lfloor \sqrt{D}\right\rfloor+\sqrt{D}$ is reduced.
    Thus the simple continued fraction of $\left\lfloor \sqrt{D}\right\rfloor+\sqrt{D}$
    is purely periodic.

    Note that the initial partial quotient of the simple continued fraction of
    $\left\lfloor \sqrt{D}\right\rfloor+\sqrt{D}$ is
    \[
        \left\lfloor\left\lfloor \sqrt{D}\right\rfloor+\sqrt{D}\right\rfloor
        =2\left\lfloor \sqrt{D}\right\rfloor
        =2a_0,
    \]
    where $a_0=\left\lfloor \sqrt{D}\right\rfloor$.
    Thus
    \begin{align*}
        \left\lfloor \sqrt{D}\right\rfloor+\sqrt{D}
        &= \left[ \overline{2a_0;\,a_1,\,a_2,\,\cdots,\,a_n} \right] \\
        &= \left[ 2a_0;\,a_1,\,a_2,\,\cdots,\,a_n,\,\overline{2a_0, \,a_1,\,a_2,\,\cdots,\,a_n} \right]
    \end{align*}
    hence
    \begin{align*}
        \sqrt{D}
        &= \left[ a_0;\,a_1,\,a_2,\,\cdots,\,a_n,\,\overline{2a_0, \,a_1,\,a_2,\,\cdots,\,a_n} \right] \\
        &= \left[ a_0;\,\overline{a_1,\,a_2,\,\cdots,\,a_n,\,2a_0} \right].
    \end{align*}
\end{remark}

\paragraph{FYI} $e=\left[ 2;\,1,\,2,\,1,\,1,\,4,\,1,\,1,\,6,\,1,\,1,\,8,\,\cdots \right]$

$\pi = \left[ 3;\,7,\,15,\,1,\,282,\,1,\,1,\,2,\,\cdots \right]$

\[
    \pi = \dfrac{4}{1+\dfrac{1^2}{2+\dfrac{3^2}{2+\dfrac{5^2}{2+\dfrac{7^2}{2+\dfrac{9^2}{\vdots}}}}}}
    = \dfrac{4}{1+\dfrac{1^2}{3+\dfrac{2^2}{5+\dfrac{3^2}{7+\dfrac{4^2}{9+\dfrac{5^2}{\vdots}}}}}}
\]
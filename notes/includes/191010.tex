\begin{proof}
    \begin{enumerate}
        \item As $x$ runs through $\frac{p-1}{2}$ even integers from 2 to $p-1$, then
        $p-x$ runs through odd integers from $p-2$ down to 1.
        Then 
        \[
            \left(2\cdot 4\cdot 6 \times \cdots \times \left(p-1\right)\right)
            \equiv \left(-1\right)^{\frac{p-1}{2}} \left(1\cdot 3\cdot 5\times \cdots \times \left(p-2\right)\right)
            \mbox{ (mod $p$)}
        \]
        and so
        \[
            \left(2\cdot 4\cdot 6 \times \cdots \times \left(p-1\right)\right)^2
            \equiv \left(1\cdot 3\cdot 5\times \cdots \times \left(p-2\right)\right)^2
            \mbox{ (mod $p$)}.
        \]

        By Wilson's theorem,
        \[
            -1 \equiv \left(p-1\right)! = \left(1 \cdot 3 \cdot 5 \times \cdots \times \left(p-2\right)\right)
            \left(2\cdot 4\cdot 6 \times \cdots \times \left(p-1\right)\right)
            \mbox{ (mod $p$)}.
        \]
        Thus
        \[
            \left(-1\right)^{\frac{p-1}{2}}\left(1\cdot 3\cdot 5\times \cdots \times \left(p-2\right)\right)^2
            \equiv -1 \mbox{ (mod $p$)},
        \]
        hence
        \[
            \left(1\cdot 3\cdot 5\times \cdots \times \left(p-2\right)\right)^2
            \equiv \left(-1\right)^{\frac{p+1}{2}} \mbox{ (mod $p$)}.
        \]
        \qed
    \item ($\Rightarrow$) Suppose $x_0^2 \equiv -1$ (mod $p$) for some $x \in \mathbb{Z}$. Then
        \[
            x_0^{p-1} = \left(x_0^2\right)^{\frac{p-1}{2}} \equiv \left(-1\right)^{\frac{p-1}{2}}.
        \]
        On the other hand, by Euler's theorem, $x_0^{p-1} \equiv 1$ (mod $p$).
        \footnote{
            Note that $x_0^2 \equiv -1$ (mod $p$), thus $\left(x_0,\,p\right) \divides 1$,
            and so $\left(x_0,\,p\right)=1$; i. e. $p \ndivides x_0$.
        }
        Thus $\left(-1\right)^{\frac{p-1}{2}} \equiv 1$ (mod $p$); i. e. 
        \[
            p \divides \left[1-\left(-1\right)^{\frac{p-1}{2}}\right].
        \]
        Hence $1-\left(-1\right)^{\frac{p-1}{2}}=0$.
        \footnote{
            If $1-\left(-1\right)^{\frac{p-1}{2}}=2\neq 0$, then $p \divides 2$. $\rightarrow\leftarrow$
        }
        Therefore $\frac{p-1}{2}$ is even and so $p \equiv 1$ (mod 4).

        ($\Leftarrow$) Note that
        \begin{align*}
            \left(p-1\right)!
            &= \left(1 \cdot 2 \cdot 3 \times \cdots \times \frac{p-1}{2}\right)
            \left(\left(p-1\right) \left(p-2\right) \left(p-3\right) \times \cdots \times \frac{p+1}{2}\right) \\
            &\equiv \left(1 \cdot 2 \cdot 3 \times \cdots \times \frac{p-1}{2}\right)
            \left(\left(-1\right) \left(-2\right) \left(-3\right) \times \cdots \times \frac{-\left(p-1\right)}{2}\right)
            \mbox{ (mod $p$)} \\
            &\equiv \left(-1\right)^{\frac{p-1}{2}} \times 1^2 \cdot 2^2 \cdot 3^2 \times \cdots \times \left(\frac{p+1}{2}\right)^2
            \mbox{ (mod $p$)} \\
        \end{align*}
        Thus
        \[
            -1 \equiv \left(p-1\right)! \equiv \left(1 \cdot 2 \cdot 3 \times \cdots \times \frac{p-1}{2}\right)^2 \mbox{ (mod $p$)}.
        \]
        Put $x_0 = 1 \cdot 2 \cdot 3 \times \cdots \times \frac{p-1}{2}$. Then $x_0^2 \equiv -1$ (mod $p$).
        \qed
    \end{enumerate}
\end{proof}

\begin{theorem}
    Let $p$ be a prime number and $e \in \mathbb{Z}^+$. Then
    \[
        \phi\left(p^e\right) = p^e - p^{e-1}.    
    \]
\end{theorem}

\begin{proof}
    Note that
    \begin{align*}
        \phi\left(p^2\right) &= \mbox{(the number of positive integers $\leq p^e$ which are relatively prime to $p^e$)} \\
        &= p^e - \mbox{(the number of positive integers $\leq p^e$ which are NOT relatively prime to $p^e$)} \\
    \end{align*}
    while the positive integers $\leq p^e$ which are NOT relatively prime to $p^e$ are
    \[
        p,\,2p,\,3p,\,\cdots,\,\left(p^{e-1}\right)p.
    \]
    \qed
\end{proof}

\begin{remark}
    \begin{enumerate}
        \item $\phi\left(p^e\right) = p^e-p^{e-1} = p^e\left(1-\frac{1}{p}\right)$.
        \item Let $n=p_1^{e_1}p_2^{e_2}\times\cdots\times p_k^{e_k}$. Then
        \begin{align*}
            \phi\left(n\right)&=  \phi\left(p_1^{e_1}\right) \phi\left(p_2^{e_2}\right)\times\cdots\times \phi\left(p_k^{e_k}\right) \\
            &= p_1^{e_1} \left(1-\frac{1}{p_1}\right) p_2^{e_2} \left(1-\frac{1}{p_2}\right) \times\cdots\times p_k^{e_k} \left(1-\frac{1}{p_k}\right) \\
            &= \left[p_1^{e_1} p_2^{e_2} \times\cdots\times p_k^{e_k}\right] \times \left[\left(1-\frac{1}{p_1}\right) \left(1-\frac{1}{p_2}\right) \times\cdots\times \left(1-\frac{1}{p_k}\right)\right] \\
            &= n \left[\left(1-\frac{1}{p_1}\right) \left(1-\frac{1}{p_2}\right) \times\cdots\times \left(1-\frac{1}{p_k}\right)\right].
        \end{align*}
    \end{enumerate}
\end{remark}

\begin{note}
    $m=4$, $n=7$ ($\left(m,\,n\right)=1$)

    $\phi\left(mn\right)=\phi\left(28\right) = 12 = 2 \times 6 =\phi\left(4\right)\phi\left(7\right)$.
\end{note}

\begin{lemma}
    If $m,\,n \in \mathbb{Z}^+,$ $r \in \mathbb{Z},$ $\left(m,\,n\right)=1,$ then
    the integers $r,\,m+r,\,2m+r,\,\cdots,\,\left(m-1\right)m+r$
    are congrufent to $0,\,1,\,2,\,\cdots,\,n-1$ modulo $n$.
\end{lemma}
\begin{proof}
    Suffies to show that no two integers in the list are congrugent modulo $n$.

    Suppose that $km+r \equiv lm+r$ (mod $n$) where $0 \leq k,\,l < n$. Then
    $km \equiv lm$ (mod $n$).
    Since $\left(m,\,n\right)=1$, hence $k \equiv l$ (mod $n$).
    Since $0 \leq k,\,l < n$, $k=l$. \qed
\end{proof}

\begin{theorem}
    $\phi\left(mn\right)=\phi\left(m\right)\phi\left(n\right)$ if $\left(m,\,n\right)=1$.
\end{theorem}
\begin{proof}
    Consider
    \begingroup
    \setlength{\arraycolsep}{10pt}
    \[
        \begin{matrix}
            1 & m+1 & 2m+1 & \cdots & \left(n-1\right)m+1 \\
            2 & m+2 & 2m+2 & \cdots & \left(n-1\right)m+2 \\
            \vdots & \vdots & \vdots & \ddots & \vdots \\
            m & 2m & 3m & \cdots & nm
        \end{matrix}
    \]
    \endgroup
    Let $r \leq m$ be a positive integer with $\left(r,\,m\right)>1$. Let $d=\left(r,\,m\right)$.
    Then $d\divides r$, $d\divides m$, and so $d \divides \left(km+r\right)$ for any $k \in \mathbb{Z}$;
    i. e. $d$ is a factor of every element in the $r$\textsuperscript{th} row.

    Thus no element in the $r$\textsuperscript{th} row is relatively prime to $m$ and hence
    to $mn$ if $\left(r,\,m\right)>1$. Hence, there are $\phi\left(m\right)$ rows
    satisfying $\left(r,\,m\right)=1$.

    Consider now the $r$\textsuperscript{th} row where $\left(r,\,m\right)=1$.
    \[
        r,\,m+r,\,2m+r,\,\cdots,\,\left(n-1\right)m+r
    \]
    By Lemma, exactly $\phi\left(n\right)$ elements in the $r$\textsuperscript{th}
    row are relatively prime to $n$, and hence to $mn$.
    Hence we conclude that $\phi\left(mn\right)=\phi\left(m\right)\phi\left(n\right)$ if $\left(m,\,n\right)=1$.
    \qed
\end{proof}

\begin{note}
    $n=28$, $d=n$. $C_d :=$ (the class of positive integers $m \leq n$ satisfying $\left(m,\,n\right)=d$). Then

    \begin{align*}
        C_1 &= \left\{1,\,3,\,5,\,9,\,11,\,13,\,15,\,17,\,19,\,23,\,25,\,27\right\} & 12&=\phi\left(28\right)=\phi\left(\frac{28}{1}\right) \\
        C_2 &= \left\{2,\,6,\,10,\,18,\,22,\,26\right\} & 6&=\phi\left(14\right)=\phi\left(\frac{28}{2}\right) \\
        C_4 &= \left\{4,\,8,\,12,\,20,\,24,\,28\right\} & 6&=\phi\left(7\right)=\phi\left(\frac{28}{4}\right) \\
        C_7 &= \left\{7,\,21\right\} & 2&=\phi\left(4\right)=\phi\left(\frac{28}{7}\right) \\
        C_{14} &= \left\{14\right\} & 1&=\phi\left(2\right)=\phi\left(\frac{28}{14}\right) \\
        C_{28} &= \left\{28\right\} & 1&=\phi\left(1\right)=\phi\left(\frac{28}{28}\right) \\
    \end{align*}
    \[
        12+6+6+2+1+1=28.
    \]
\end{note}

\begin{theorem}
    For $n \in \mathbb{Z}^+$,
    \[
        n = \sum_{d \divides n} \phi\left(d\right) = \sum_{d \divides n} \phi\left(\frac{n}{d}\right).
    \]
\end{theorem}

\begin{proof}
    Let $m \in \mathbb{Z}^+$ such that $m \leq n$.
    Then $m \in C_d$ if and only if $\left(m,\,n\right)=d$, if and only if $\left(\frac{m}{d},\,\frac{n}{d}\right)=1$.

    Thus the number of positive integers $\leq\frac{n}{d}$ which are relatively prime to $\frac{n}{d}$
    is equlal to the number og elements $m$ in $C_d$.
    Hence each class $C_d$ has $\phi\left(\frac{n}{d}\right)$ elements.

    Since there is a class corresponding to elery factor $d$ of $n$ and every integer $m\leq n$ belongs to
    exactly one class, it follows that the sum of the count of elements in various classes is $n$;
    i. e. $\sum_{d \divides n} \phi\left(\frac{n}{d}\right)=n$.

    As $d$ runs over the divisors of $n$, so does $\frac{n}{d}$.
    Hence $\sum_{d \divides n} \phi\left(d\right)=n$.
    \qed 
\end{proof}

\begin{theorem}[Chinese Remainder Theorem]
    Let $m_1,\,m_2,\,\cdots,\,m_r$ are pairwise relatively prime positive integers.
    Then the system of congrugences
    \[
        \begin{cases}
            x\equiv a_1 & \mbox{(mod $m_1$)} \\
            x\equiv a_2 & \mbox{(mod $m_2$)} \\
            & \vdots \\
            x\equiv a_r & \mbox{(mod $m_r$)} \\
        \end{cases}
    \]
    where $a_i \in \mathbb{Z}$, has a unique solution modulo $M=m_1m_2\times\cdots\times m_r$.
\end{theorem}

\begin{proof}
    (Proof left for homework -- due October 15th.)
\end{proof}

\begin{example}
    \begin{enumerate}
        \item \[
            \begin{cases}
                x \equiv 1 & \mbox{(mod 4)} \\
                x \equiv 3 & \mbox{(mod 5)} \\
                x \equiv 2 & \mbox{(mod 7)}
            \end{cases}    
        \]
        \begin{align*}
            35 \times \underline{?}_\text{3} &\equiv 1 & \mbox{(mod 4)}\\
            28 \times \underline{?}_\text{2} &\equiv 3 & \mbox{(mod 5)}\\
            20 \times \underline{?}_\text{6} &\equiv 2 & \mbox{(mod 7)}
        \end{align*}
        
        Note that
        \begin{align*}
            M=4\cdot 5\cdot 7 &= 35 \cdot 4 = M_1m_1 \\
            &= 28 \cdot 5 = M_2m_2 \\
            &= 20 \cdot 7 = M_3m_3
        \end{align*}
        thus $x=1\cdot 35\cdot 3 + 3\cdot 28\cdot 2 + 2 \cdot 20 \cdot 6 = 93$ (mod 140)
        \item \begin{align*}
            & \begin{cases}
                8x \equiv 4 & \mbox{(mod 14)} \\
                5x \equiv 3 & \mbox{(mod 11)} \\
            \end{cases} \\
            & \Leftrightarrow \begin{cases}
                4x \equiv 2 & \mbox{(mod 7)} \\
                5x \equiv 3 & \mbox{(mod 11)} \\
            \end{cases} \\
            & \Leftrightarrow \begin{cases}
                x \equiv 4 & \mbox{(mod 7)} \\
                x \equiv 5 & \mbox{(mod 11)} \\
            \end{cases}
        \end{align*}

        By CRT, $x=4 \cdot 11 \cdot 2 + 5 \cdot 7 \cdot 8 \equiv 368 \equiv 60$ (mod 77)

        Note that $x \equiv 60$ (mod 77) $\Leftrightarrow$ $x \equiv 60$, $x \equiv 137$ (mod 154).
    \end{enumerate}
\end{example}

(Just returned from Bangkok; due to jet lag, errors may occur for this day's lecture notes. Please read with care.)

\paragraph{Question.} How can we calcaulate $\legendre{a}{p}$?

Suppose that $a=\pm p_1^{e_1}p_2^{e_2}\times\cdots p_i^{e_i}$, where 
$p_1,\,p_2,\,p_3,\,\cdots,\,p_i$ are distinct primes.
Since $p \ndivides a$ and $p \neq p_i$, $\forall 1 \leq i \leq k$.
Thus
\[
    \legendre{a}{p} = \legendre{\pm 1}{p}\legendre{p_1}{p}^{e_1}\times\cdots\times\legendre{p_i}{p}^{e_i}.
\] 

Hence in order to calcaulate $\legendre{a}{p}$, it suffice to be able to calculate $\legendre{q}{p}$
where $p$ and $q$ are distinct primes.

\begin{lemma}[Gauss's Lemma]
    $p$ : odd prime. $p \ndivides a$.

    If $s$ is the number of least positive residues mod $p$ of the integers
    $a,\,2a,\,3a,\,\cdots,\,\frac{p-1}{2}a$ that are greater than $\frac{p}{2}$, then
    $\legendre{a}{p}=\left(-1\right)^s$.
\end{lemma}

e. g. $\legendre{5}{11}$

The least positive residues of
\[
    1\cdot 5 \quad 2 \cdot 5 \quad 3 \cdot 5 \quad 4 \cdot 5 \quad 5 \cdot 5
\]
Note that $s=2$.

\begin{proof}
    (Proof left for homework; due November 7th.)
\end{proof}

\begin{theorem}
    $p$ : odd prime.

    \begin{align*}
        \legendre{2}{p}&=\left(-1\right)^{\frac{p^2-1}{8}} \\
        &= \begin{cases}
            1 & \mbox{if $p \equiv \pm 1 \pmod{8}$} \\
            -1 & \mbox{if $p \equiv \pm 3 \pmod{8}$}
        \end{cases}
    \end{align*}
    \footnote{$\star$}
\end{theorem}

\begin{proof}
    By Gauss's Lemma, $\legendre{2}{p}=\left(-1\right)^s$
    where $s$ is the number of least positive residues mod $p$ of the integers $1\cdot 2$,
    $2 \cdot 2$, $3 \cdot 2$, $\cdots$, $\frac{p-1}{2}\times 2$ that are greater than $\frac{p}{2}$.  
    
    Since all of these integers are less than $p$, we need only count these greater than $\frac{p}{2}$ to find.
    Note that the integers $2j$ where $1\leq j\leq \frac{p-1}{2}$ are less than $\frac{p}{2}$ whem
    $j \leq \frac{p}{4}$.

    Thus $\exists \left\lfloor\frac{p}{4}\right\rfloor$ integers in the set less
    that $\frac{p}{2}$. By Gauss's lemma,
    \[
        \legendre{2}{p} = \left(-1\right)^{\frac{p-1}{2} - \left\lfloor\frac{p}{4}\right\rfloor}.
    \]
    To prove the theorem, it suffice to show that $\frac{p-1}{2} - \left\lfloor\frac{p}{4}\right\rfloor \equiv \frac{p^2-1}{8} \pmod{2}$
    for every odd integer $p$.

    Note that it holds for a positive integer $p$ if and only if it holds for $p+8$.
    It can be checked that it holds for $p\equiv\pm 1$, $p\equiv\pm 3 \pmod{8}$.
    Hence we conclude that it holds for every odd integer $p$.
\end{proof}

e. g. 
\[
    \legendre{89}{13}=\legendre{-2}{13}=\legendre{-1}{13}\legendre{2}{13}
    = 1\left(-1\right) =-1
\]

\begin{theorem}[The Law of Quadratic Reciprocity]
    $p,\,q$ : distinct odd primes.
    \[
        \legendre{p}{q}=\left(-1\right)^{\frac{p-1}{2}\frac{q-1}{2}}\legendre{q}{p}
    \]
    or
    \[
        \legendre{p}{q}\legendre{q}{p} = \left(-1\right)^{\frac{p-1}{2}\frac{q-1}{2}}
    \]
\end{theorem}

\begin{proof}
    (See the book.)
\end{proof}

\begin{remark}
    Note that
    \[
        \frac{p-1}{2} = \begin{cases}
            \mbox{even} & \mbox{if $p\equiv 1 \pmod{4}$} \\
            \mbox{odd} & \mbox{if $p\equiv 3 \pmod{4}$}
        \end{cases}
    \]
    thus
    \[
        \frac{p-1}{2}\frac{q-1}{2} = \begin{cases}
            \mbox{even} & \mbox{if $p\equiv 1$ or $q\equiv 1 \pmod{4}$} \\
            \mbox{odd} & \mbox{if $p\equiv q\equiv 3 \pmod{4}$}
        \end{cases}
    \]
    hence 
    \[
        \legendre{p}{q}\legendre{q}{p} = \begin{cases}
            1 & \mbox{if $p\equiv 1$ or $q\equiv 1 \pmod{4}$} \\
            -1 & \mbox{if $p\equiv q\equiv 3 \pmod{4}$}
        \end{cases}
    \]
    Since $\legendre{p}{q}$ and $\legendre{q}{p}$ are $\pm 1$,
    \[
        \legendre{p}{q}\legendre{q}{p} = \begin{cases}
            \legendre{q}{p} & \mbox{if $p\equiv 1$ or $q\equiv 1 \pmod{4}$} \\
            -\legendre{q}{p} & \mbox{if $p\equiv q\equiv 3 \pmod{4}$}
        \end{cases}
    \]
\end{remark}

\begin{theorem}
    $p,\,q=2p+1$ : $p$, $q$ are odd primes.

    $M_p=2^p-1$ : Mersenne number.

    Then one of the following holds.
    \begin{enumerate}
        \item $q \divides M_p$
        \item $q \divides \left( M_p + 2 \right)$
    \end{enumerate}

    Moreover,
    \begin{align*}
        q \divides M_p & \Leftrightarrow q \equiv 1 \mbox{ or } q \equiv -3 \pmod{8} \\
        q \divides \left(M_p + 2\right) & \Leftrightarrow q \equiv -1 \mbox{ or } q \equiv 3 \pmod{8}
    \end{align*}
\end{theorem}

\begin{proof}
    By Fermat's, $2^{q-1} \equiv 0 \pmod{q}$, then
    \begin{align*}
        M_p \left(M_p+2\right) &= \left(2^p-1\right)\left(2^p+1\right) \\
        &= 2^{2p}-1 = 2^{q-1}-1 \\
        &\equiv 0 \pmod{q}.
    \end{align*}
\end{proof}

Since $q$ is prime, $q\divides M_p$ or $q\divides \left(M_p+2\right)$.

Note that
\begin{align*}
    q \divides M_p &\Leftrightarrow 2^p = 2^{\frac{q-1}{2}} \equiv 1 \pmod{q} \\
    & \Leftrightarrow \legendre{2}{q}=1
\end{align*}
and
\begin{align*}
    2 \divides \left(2^p-2\right) &\Leftrightarrow 2^p = 2^{\frac{q-1}{2}} \equiv -1 \pmod{q} \\
    &\Leftrightarrow \legendre{2}{q}=-1
\end{align*}

Hence it follows from the fact that
\[
    \legendre{2}{q} = \left(-1\right)^{\frac{q^2-1}{8}} = \begin{cases}
        1 & q\equiv \pm 1 \pmod{8} \\
        -1 & q\equiv \pm 3 \pmod{8}
    \end{cases}
\]

\begin{theorem}[Pelpin's Test]
    $F_m=2^{2^m}+1$ is a prime if and only if $3^{\frac{F_m-1}{2}} \equiv -1 \pmod{F_m}$.
\end{theorem}

\begin{proof}
    ($\Leftarrow$) Suppose that $3^{\frac{F_m-1}{2}}\equiv -1\pmod{F_m}$.
    Then $3^{F_m-1} \equiv 1 \pmod{F_m}$.

    If $p$ is a prime that $p\divides F_m$, then $3^{F_m-1} \equiv 1 \pmod{p}$.
    Note that $\left(3,\,p\right)=1$. Then \[\ord_p 3 \divides \left(F_m-1\right) = 2^{2^m}\],
    and so $\ord_p 3$ must be a power of 2.

    However, since $\ord_p 3 \ndivides 2^{2^n-1} = \frac{F_m-1}{2}$. This is a contradiction.
\end{proof}
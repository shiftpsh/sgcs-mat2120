Using the Euclidean algorithm, we can express rational numbers
as continued fractions.

\begin{definition}[Finite Continued Fraction]
    A \textbf{finite continued fraction} is an expression of the form
    \[
        a_0 + \dfrac{1}{a_1 + \dfrac{1}{a_2 + \dfrac{1}{\ddots + \dfrac{1}{a_n}}}}
        = \left[a_0;\,a_1,\,a_2,\,\cdots,\,a_n\right]
    \]
    where $a_0,\,a_1,\,\cdots,\,a_n$ are real numbers whom of which
    $a_1,\,a_2,\,\cdots,\,a_n$ are positive.
    
    A finite continued fraction is simple if $a_0,\,a_1,\,\cdots,\,a_n \in \mathbb{Z}$.
\end{definition}

\begin{theorem}
    Every finite simple continued fraction represents a rational number.
    Conversely, every rational number can be expressed as a finite simple
    continued fraction.
\end{theorem}

\begin{remark}
    $\frac{62}{23} = \left[2,\,1,\,2,\,3,\,2\right] = \left[2,\,1,\,2,\,3,\,1,\,1\right]$
\end{remark}

\begin{definition}
    A continued fraction $\left[a_0;\,a_1,\,\cdots,\,a_k\right]$
    where $k$ is a nonnegative integer less than $n$ is called the
    \textbf{$k$-th convergerent} of a continued fraction $\left[a_0;\,a_1,\,\cdots,\,a_n\right]$.

    The $k$-th convergent is denoted by $C_k$.
\end{definition}

\begin{theorem}
    Let $a_0,\,a_1,\,\cdots,\,a_n$ be real numbers with $a_1,\, a_2,\,\cdots,\, a_n > 0$.
    Define the sequence $\left\{p_i\right\}$ and $\left\{q_i\right\}$ 
    recursively as
    \[
        \begin{cases}
            p_0 := a_0 & q_0 := 1 \\
            p_1 := a_1a_0 & q_1 := a_1 \\
            p_k := a_kp_{k-1} + p_{k-2} & q_k := a_k q_{k-1} + q_{k-2} \qquad \mbox{ for } k = 2,\,3,\,\cdots,\,n
        \end{cases}
    \]
    then $C_k = \frac{p_k}{q_k}$.
\end{theorem}